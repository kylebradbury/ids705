% Options for packages loaded elsewhere
\PassOptionsToPackage{unicode}{hyperref}
\PassOptionsToPackage{hyphens}{url}
\PassOptionsToPackage{dvipsnames,svgnames,x11names}{xcolor}
%
\documentclass[
  letterpaper,
  DIV=11,
  numbers=noendperiod]{scrartcl}

\usepackage{amsmath,amssymb}
\usepackage{iftex}
\ifPDFTeX
  \usepackage[T1]{fontenc}
  \usepackage[utf8]{inputenc}
  \usepackage{textcomp} % provide euro and other symbols
\else % if luatex or xetex
  \usepackage{unicode-math}
  \defaultfontfeatures{Scale=MatchLowercase}
  \defaultfontfeatures[\rmfamily]{Ligatures=TeX,Scale=1}
\fi
\usepackage{lmodern}
\ifPDFTeX\else  
    % xetex/luatex font selection
\fi
% Use upquote if available, for straight quotes in verbatim environments
\IfFileExists{upquote.sty}{\usepackage{upquote}}{}
\IfFileExists{microtype.sty}{% use microtype if available
  \usepackage[]{microtype}
  \UseMicrotypeSet[protrusion]{basicmath} % disable protrusion for tt fonts
}{}
\makeatletter
\@ifundefined{KOMAClassName}{% if non-KOMA class
  \IfFileExists{parskip.sty}{%
    \usepackage{parskip}
  }{% else
    \setlength{\parindent}{0pt}
    \setlength{\parskip}{6pt plus 2pt minus 1pt}}
}{% if KOMA class
  \KOMAoptions{parskip=half}}
\makeatother
\usepackage{xcolor}
\setlength{\emergencystretch}{3em} % prevent overfull lines
\setcounter{secnumdepth}{-\maxdimen} % remove section numbering
% Make \paragraph and \subparagraph free-standing
\makeatletter
\ifx\paragraph\undefined\else
  \let\oldparagraph\paragraph
  \renewcommand{\paragraph}{
    \@ifstar
      \xxxParagraphStar
      \xxxParagraphNoStar
  }
  \newcommand{\xxxParagraphStar}[1]{\oldparagraph*{#1}\mbox{}}
  \newcommand{\xxxParagraphNoStar}[1]{\oldparagraph{#1}\mbox{}}
\fi
\ifx\subparagraph\undefined\else
  \let\oldsubparagraph\subparagraph
  \renewcommand{\subparagraph}{
    \@ifstar
      \xxxSubParagraphStar
      \xxxSubParagraphNoStar
  }
  \newcommand{\xxxSubParagraphStar}[1]{\oldsubparagraph*{#1}\mbox{}}
  \newcommand{\xxxSubParagraphNoStar}[1]{\oldsubparagraph{#1}\mbox{}}
\fi
\makeatother

\usepackage{color}
\usepackage{fancyvrb}
\newcommand{\VerbBar}{|}
\newcommand{\VERB}{\Verb[commandchars=\\\{\}]}
\DefineVerbatimEnvironment{Highlighting}{Verbatim}{commandchars=\\\{\}}
% Add ',fontsize=\small' for more characters per line
\usepackage{framed}
\definecolor{shadecolor}{RGB}{241,243,245}
\newenvironment{Shaded}{\begin{snugshade}}{\end{snugshade}}
\newcommand{\AlertTok}[1]{\textcolor[rgb]{0.68,0.00,0.00}{#1}}
\newcommand{\AnnotationTok}[1]{\textcolor[rgb]{0.37,0.37,0.37}{#1}}
\newcommand{\AttributeTok}[1]{\textcolor[rgb]{0.40,0.45,0.13}{#1}}
\newcommand{\BaseNTok}[1]{\textcolor[rgb]{0.68,0.00,0.00}{#1}}
\newcommand{\BuiltInTok}[1]{\textcolor[rgb]{0.00,0.23,0.31}{#1}}
\newcommand{\CharTok}[1]{\textcolor[rgb]{0.13,0.47,0.30}{#1}}
\newcommand{\CommentTok}[1]{\textcolor[rgb]{0.37,0.37,0.37}{#1}}
\newcommand{\CommentVarTok}[1]{\textcolor[rgb]{0.37,0.37,0.37}{\textit{#1}}}
\newcommand{\ConstantTok}[1]{\textcolor[rgb]{0.56,0.35,0.01}{#1}}
\newcommand{\ControlFlowTok}[1]{\textcolor[rgb]{0.00,0.23,0.31}{\textbf{#1}}}
\newcommand{\DataTypeTok}[1]{\textcolor[rgb]{0.68,0.00,0.00}{#1}}
\newcommand{\DecValTok}[1]{\textcolor[rgb]{0.68,0.00,0.00}{#1}}
\newcommand{\DocumentationTok}[1]{\textcolor[rgb]{0.37,0.37,0.37}{\textit{#1}}}
\newcommand{\ErrorTok}[1]{\textcolor[rgb]{0.68,0.00,0.00}{#1}}
\newcommand{\ExtensionTok}[1]{\textcolor[rgb]{0.00,0.23,0.31}{#1}}
\newcommand{\FloatTok}[1]{\textcolor[rgb]{0.68,0.00,0.00}{#1}}
\newcommand{\FunctionTok}[1]{\textcolor[rgb]{0.28,0.35,0.67}{#1}}
\newcommand{\ImportTok}[1]{\textcolor[rgb]{0.00,0.46,0.62}{#1}}
\newcommand{\InformationTok}[1]{\textcolor[rgb]{0.37,0.37,0.37}{#1}}
\newcommand{\KeywordTok}[1]{\textcolor[rgb]{0.00,0.23,0.31}{\textbf{#1}}}
\newcommand{\NormalTok}[1]{\textcolor[rgb]{0.00,0.23,0.31}{#1}}
\newcommand{\OperatorTok}[1]{\textcolor[rgb]{0.37,0.37,0.37}{#1}}
\newcommand{\OtherTok}[1]{\textcolor[rgb]{0.00,0.23,0.31}{#1}}
\newcommand{\PreprocessorTok}[1]{\textcolor[rgb]{0.68,0.00,0.00}{#1}}
\newcommand{\RegionMarkerTok}[1]{\textcolor[rgb]{0.00,0.23,0.31}{#1}}
\newcommand{\SpecialCharTok}[1]{\textcolor[rgb]{0.37,0.37,0.37}{#1}}
\newcommand{\SpecialStringTok}[1]{\textcolor[rgb]{0.13,0.47,0.30}{#1}}
\newcommand{\StringTok}[1]{\textcolor[rgb]{0.13,0.47,0.30}{#1}}
\newcommand{\VariableTok}[1]{\textcolor[rgb]{0.07,0.07,0.07}{#1}}
\newcommand{\VerbatimStringTok}[1]{\textcolor[rgb]{0.13,0.47,0.30}{#1}}
\newcommand{\WarningTok}[1]{\textcolor[rgb]{0.37,0.37,0.37}{\textit{#1}}}

\providecommand{\tightlist}{%
  \setlength{\itemsep}{0pt}\setlength{\parskip}{0pt}}\usepackage{longtable,booktabs,array}
\usepackage{calc} % for calculating minipage widths
% Correct order of tables after \paragraph or \subparagraph
\usepackage{etoolbox}
\makeatletter
\patchcmd\longtable{\par}{\if@noskipsec\mbox{}\fi\par}{}{}
\makeatother
% Allow footnotes in longtable head/foot
\IfFileExists{footnotehyper.sty}{\usepackage{footnotehyper}}{\usepackage{footnote}}
\makesavenoteenv{longtable}
\usepackage{graphicx}
\makeatletter
\newsavebox\pandoc@box
\newcommand*\pandocbounded[1]{% scales image to fit in text height/width
  \sbox\pandoc@box{#1}%
  \Gscale@div\@tempa{\textheight}{\dimexpr\ht\pandoc@box+\dp\pandoc@box\relax}%
  \Gscale@div\@tempb{\linewidth}{\wd\pandoc@box}%
  \ifdim\@tempb\p@<\@tempa\p@\let\@tempa\@tempb\fi% select the smaller of both
  \ifdim\@tempa\p@<\p@\scalebox{\@tempa}{\usebox\pandoc@box}%
  \else\usebox{\pandoc@box}%
  \fi%
}
% Set default figure placement to htbp
\def\fps@figure{htbp}
\makeatother

\KOMAoption{captions}{tableheading}
\makeatletter
\@ifpackageloaded{tcolorbox}{}{\usepackage[skins,breakable]{tcolorbox}}
\@ifpackageloaded{fontawesome5}{}{\usepackage{fontawesome5}}
\definecolor{quarto-callout-color}{HTML}{909090}
\definecolor{quarto-callout-note-color}{HTML}{0758E5}
\definecolor{quarto-callout-important-color}{HTML}{CC1914}
\definecolor{quarto-callout-warning-color}{HTML}{EB9113}
\definecolor{quarto-callout-tip-color}{HTML}{00A047}
\definecolor{quarto-callout-caution-color}{HTML}{FC5300}
\definecolor{quarto-callout-color-frame}{HTML}{acacac}
\definecolor{quarto-callout-note-color-frame}{HTML}{4582ec}
\definecolor{quarto-callout-important-color-frame}{HTML}{d9534f}
\definecolor{quarto-callout-warning-color-frame}{HTML}{f0ad4e}
\definecolor{quarto-callout-tip-color-frame}{HTML}{02b875}
\definecolor{quarto-callout-caution-color-frame}{HTML}{fd7e14}
\makeatother
\makeatletter
\@ifpackageloaded{caption}{}{\usepackage{caption}}
\AtBeginDocument{%
\ifdefined\contentsname
  \renewcommand*\contentsname{Table of contents}
\else
  \newcommand\contentsname{Table of contents}
\fi
\ifdefined\listfigurename
  \renewcommand*\listfigurename{List of Figures}
\else
  \newcommand\listfigurename{List of Figures}
\fi
\ifdefined\listtablename
  \renewcommand*\listtablename{List of Tables}
\else
  \newcommand\listtablename{List of Tables}
\fi
\ifdefined\figurename
  \renewcommand*\figurename{Figure}
\else
  \newcommand\figurename{Figure}
\fi
\ifdefined\tablename
  \renewcommand*\tablename{Table}
\else
  \newcommand\tablename{Table}
\fi
}
\@ifpackageloaded{float}{}{\usepackage{float}}
\floatstyle{ruled}
\@ifundefined{c@chapter}{\newfloat{codelisting}{h}{lop}}{\newfloat{codelisting}{h}{lop}[chapter]}
\floatname{codelisting}{Listing}
\newcommand*\listoflistings{\listof{codelisting}{List of Listings}}
\makeatother
\makeatletter
\makeatother
\makeatletter
\@ifpackageloaded{caption}{}{\usepackage{caption}}
\@ifpackageloaded{subcaption}{}{\usepackage{subcaption}}
\makeatother

\usepackage{bookmark}

\IfFileExists{xurl.sty}{\usepackage{xurl}}{} % add URL line breaks if available
\urlstyle{same} % disable monospaced font for URLs
\hypersetup{
  pdftitle={Assignment 2},
  pdfauthor={Kyle Bradbury},
  colorlinks=true,
  linkcolor={blue},
  filecolor={Maroon},
  citecolor={Blue},
  urlcolor={Blue},
  pdfcreator={LaTeX via pandoc}}


\title{Assignment 2}
\usepackage{etoolbox}
\makeatletter
\providecommand{\subtitle}[1]{% add subtitle to \maketitle
  \apptocmd{\@title}{\par {\large #1 \par}}{}{}
}
\makeatother
\subtitle{Supervised Machine Learning Fundamentals}
\author{Kyle Bradbury}
\date{2025-01-22}

\begin{document}
\maketitle

\renewcommand*\contentsname{Table of contents}
{
\hypersetup{linkcolor=}
\setcounter{tocdepth}{1}
\tableofcontents
}

\subsection{Instructions}\label{instructions}

\emph{Instructions for all assignments can be found
\href{https://kylebradbury.github.io/ids705/notebooks/assignment_instructions.html}{here}.
Note: this assignment falls under collaboration Mode 2: Individual
Assignment -- Collaboration Permitted. Please refer to the syllabus for
additional information. Please be sure to list the names of any students
that you worked with on this assignment. Total points in the assignment
add up to 90; an additional 10 points are allocated to professionalism
and presentation quality.}

\subsection{Learning Objectives:}\label{learning-objectives}

By successfully completing this assignment you will be able to\ldots{} -
Explain the bias-variance tradeoff of supervised machine learning and
the impact of model flexibility on algorithm performance - Perform
supervised machine learning training and performance evaluation -
Implement a k-nearest neighbors machine learning algorithm from scratch
in a style similar to that of popular machine learning tools like
\texttt{scikit-learn} - Describe how KNN classification works, the
method's reliance on distance measurements, and the impact of higher
dimensionality on computational speed - Apply regression (linear
regression) and classification (KNN) supervised learning techniques to
data and evaluate the performance of those methods - Construct simple
feature transformations for improving model fit in linear models - Fit a
\texttt{scikit-learn} supervised learning technique to training data and
make predictions using it

\subsection{Exercise 1 - Conceptual Questions on Supervised Learning
I}\label{exercise-1---conceptual-questions-on-supervised-learning-i}

\textbf{{[}4 points{]}}

For each part below, indicate whether we would generally expect the
performance of a flexible statistical learning method to be
\emph{better} or \emph{worse} than an inflexible method. Justify your
answer.

\textbf{1.1.} The sample size \(n\) is extremely large, and the number
of predictors \(p\) is small.

\textbf{1.2.} The number of predictors \(p\) is extremely large, and the
number of observations \(n\) is small.

\textbf{1.3.} The relationship between the predictors and response is
highly non-linear.

\textbf{1.4.} The variance of the error terms,
i.e.~\(\sigma^2 = Var(\epsilon)\), is extremely high.

\subsection{Exercise 2 - Conceptual Questions on Supervised Learning
II}\label{exercise-2---conceptual-questions-on-supervised-learning-ii}

\textbf{{[}6 points{]}}

For each of the following, (i) explain if each scenario is a
classification or regression problem AND why, (ii) indicate whether we
are most interested in inference or prediction for that problem AND why,
and (iii) provide the sample size \(n\) and number of predictors \(p\)
indicated for each scenario.

\textbf{2.1.} We collect a set of data on the top 500 firms in the US.
For each firm we record profit, number of employees, industry and the
CEO salary. We are interested in understanding which factors affect CEO
salary.

\textbf{2.2.} We are considering launching a new product and wish to
know whether it will be a success or a failure. We collect data on 20
similar products that were previously launched. For each product we have
recorded whether it was a success or failure, price charged for the
product, marketing budget, competition price, and ten other variables.

\textbf{2.3.} We are interested in predicting the \% change in the US
dollar in relation to the weekly changes in the world stock markets.
Hence we collect weekly data for all of 2012. For each week we record
the \% change in the dollar, the \% change in the US market, the \%
change in the British market, and the \% change in the German market.

\subsection{Exercise 3 - Classification using
KNN}\label{exercise-3---classification-using-knn}

\textbf{{[}6 points{]}}

The table below provides a training dataset containing six observations
(a.k.a. samples) (\(n=6\)) each with three predictors (a.k.a. features)
(\(p=3\)), and one qualitative response variable (a.k.a. target).

\emph{Table 1. Training dataset with \(n=6\) observations in \(p=3\)
dimensions with a categorical response, \(y\)}

\begin{longtable}[]{@{}lllll@{}}
\toprule\noalign{}
Obs. & \(x_1\) & \(x_2\) & \(x_3\) & \(y\) \\
\midrule\noalign{}
\endhead
\bottomrule\noalign{}
\endlastfoot
\textbf{1} & 0 & 3 & 0 & Red \\
\textbf{2} & 2 & 0 & 0 & Red \\
\textbf{3} & 0 & 1 & 3 & Red \\
\textbf{4} & 0 & 1 & 2 & Blue \\
\textbf{5} & -1 & 0 & 1 & Blue \\
\textbf{6} & 1 & 1 & 1 & Red \\
\end{longtable}

We want to use the above training dataset to make a prediction,
\(\hat{y}\), for an unlabeled test data observation where
\(x_1=x_2=x_3=0\) using \(K\)-nearest neighbors. You are given some code
below to get you started. \emph{Note: coding is only required for part
(a), for (b)-(d) please provide your reasoning based on your answer to
part (a)}.

\textbf{3.1.} Compute the Euclidean distance between each observation
and the test point, \(x_1=x_2=x_3=0\). Present your answer in a table
similar in style to Table 1 with observations 1-6 as the row headers.

\textbf{3.2.} What is our prediction, \(\hat{y}\), when \(K=1\) for the
test point? Why?

\textbf{3.3.} What is our prediction, \(\hat{y}\), when \(K=3\) for the
test point? Why?

\textbf{3.4.} If the Bayes decision boundary (the optimal decision
boundary) in this problem is highly nonlinear, then would we expect the
\emph{best} value of \(K\) to be large or small? Why?

\begin{Shaded}
\begin{Highlighting}[]
\ImportTok{import}\NormalTok{ numpy }\ImportTok{as}\NormalTok{ np}

\NormalTok{X }\OperatorTok{=}\NormalTok{ np.array([[ }\DecValTok{0}\NormalTok{, }\DecValTok{3}\NormalTok{, }\DecValTok{0}\NormalTok{],}
\NormalTok{              [ }\DecValTok{2}\NormalTok{, }\DecValTok{0}\NormalTok{, }\DecValTok{0}\NormalTok{],}
\NormalTok{              [ }\DecValTok{0}\NormalTok{, }\DecValTok{1}\NormalTok{, }\DecValTok{3}\NormalTok{],}
\NormalTok{              [ }\DecValTok{0}\NormalTok{, }\DecValTok{1}\NormalTok{, }\DecValTok{2}\NormalTok{],}
\NormalTok{              [}\OperatorTok{{-}}\DecValTok{1}\NormalTok{, }\DecValTok{0}\NormalTok{, }\DecValTok{1}\NormalTok{],}
\NormalTok{              [ }\DecValTok{1}\NormalTok{, }\DecValTok{1}\NormalTok{, }\DecValTok{1}\NormalTok{]])}
\NormalTok{y }\OperatorTok{=}\NormalTok{ np.array([}\StringTok{\textquotesingle{}r\textquotesingle{}}\NormalTok{,}\StringTok{\textquotesingle{}r\textquotesingle{}}\NormalTok{,}\StringTok{\textquotesingle{}r\textquotesingle{}}\NormalTok{,}\StringTok{\textquotesingle{}b\textquotesingle{}}\NormalTok{,}\StringTok{\textquotesingle{}b\textquotesingle{}}\NormalTok{,}\StringTok{\textquotesingle{}r\textquotesingle{}}\NormalTok{])}
\end{Highlighting}
\end{Shaded}

\subsection{Exercise 4 - Build your own classification
algorithm}\label{exercise-4---build-your-own-classification-algorithm}

\textbf{{[}18 points{]}}

\begin{tcolorbox}[enhanced jigsaw, opacityback=0, colframe=quarto-callout-note-color-frame, leftrule=.75mm, toptitle=1mm, colbacktitle=quarto-callout-note-color!10!white, bottomrule=.15mm, colback=white, coltitle=black, toprule=.15mm, opacitybacktitle=0.6, titlerule=0mm, arc=.35mm, bottomtitle=1mm, title=\textcolor{quarto-callout-note-color}{\faInfo}\hspace{0.5em}{Note}, rightrule=.15mm, breakable, left=2mm]

\href{https://github.com/kylebradbury/ids705/tree/main/notebooks/data/a1}{Data
for this exercise can be downloaded here}

\end{tcolorbox}

\textbf{4.1.} Build a working version of a binary KNN classifier using
the skeleton code below. We'll use the \texttt{sklearn} convention that
a supervised learning algorithm has the methods \texttt{fit} which
trains your algorithm (for KNN that means storing the data) and
\texttt{predict} which identifies the K nearest neighbors and determines
the most common class among those K neighbors. \emph{Note: Most
classification algorithms typically also have a method
\texttt{predict\_proba} which outputs the confidence score of each
prediction, but we will explore that in a later assignment. Please use
\texttt{NumPy} to implement euclidean distance function.}

\textbf{4.2.} Load the datasets to be evaluated here. Each includes
training features (\(\mathbf{X}\)), and test features (\(\mathbf{y}\))
for both a low dimensional dataset (\(p = 2\) features/predictors) and a
higher dimensional dataset (\(p = 100\) features/predictors). For each
of these datasets there are \(n=1000\) observations of each. They can be
found in the data subfolder on github (see link above). Each file is
labeled similar to \texttt{A2\_Q4\_X\_train\_low.csv}, which lets you
know whether the dataset is of features, \(X\), targets, \(y\); training
or testing; and low or high dimensions.

\textbf{4.3.} Train your classifier on first the low dimensional dataset
and then the high dimensional dataset with \(k=5\). Evaluate the
classification performance on the corresponding test data for each of
those trained models. Calculate the time it takes each model to make the
predictions and the overall accuracy of those predictions for each
corresponding set of test data - state each.

\textbf{4.4.} Compare your implementation's accuracy and computation
time to the scikit learn
\href{http://scikit-learn.org/stable/modules/generated/sklearn.neighbors.KNeighborsClassifier.html}{KNeighborsClassifier}
class. How do the results and speed compare to your implementation?
\emph{Hint: your results should be identical to that of the scikit-learn
implementation.}

\textbf{4.5.} Some supervised learning algorithms are more
computationally intensive during training than testing. What are the
drawbacks of the prediction process being slow? In what cases in
practice might slow testing (inference) be more problematic than slow
training?

\begin{Shaded}
\begin{Highlighting}[]
\CommentTok{\# Skeleton code for part (a) to write your own kNN classifier}

\KeywordTok{class}\NormalTok{ Knn:}
\CommentTok{\# k{-}Nearest Neighbor class object for classification training and testing}
    \KeywordTok{def} \FunctionTok{\_\_init\_\_}\NormalTok{(}\VariableTok{self}\NormalTok{):}
        
    \KeywordTok{def}\NormalTok{ fit(}\VariableTok{self}\NormalTok{, x, y):}
        \CommentTok{\# Save the training data to properties of this class}
        
    \KeywordTok{def}\NormalTok{ predict(}\VariableTok{self}\NormalTok{, x, k):}
\NormalTok{        y\_hat }\OperatorTok{=}\NormalTok{ [] }\CommentTok{\# Variable to store the estimated class label for }
        \CommentTok{\# Calculate the distance from each vector in x to the training data}
        
        \CommentTok{\# Return the estimated targets}
        \ControlFlowTok{return}\NormalTok{ y\_hat}

\CommentTok{\# Metric of overall classification accuracy}
\CommentTok{\#  (a more general function, sklearn.metrics.accuracy\_score, is also available)}
\KeywordTok{def}\NormalTok{ accuracy(y,y\_hat):}
\NormalTok{    nvalues }\OperatorTok{=} \BuiltInTok{len}\NormalTok{(y)}
\NormalTok{    accuracy }\OperatorTok{=} \BuiltInTok{sum}\NormalTok{(y }\OperatorTok{==}\NormalTok{ y\_hat) }\OperatorTok{/}\NormalTok{ nvalues}
    \ControlFlowTok{return}\NormalTok{ accuracy}
\end{Highlighting}
\end{Shaded}

\subsection{Exercise 5 - Bias-variance tradeoff: exploring the tradeoff
with a KNN
classifier}\label{exercise-5---bias-variance-tradeoff-exploring-the-tradeoff-with-a-knn-classifier}

\textbf{{[}20 points{]}}

This exercise will illustrate the impact of the bias-variance tradeoff
on classifier performance by investigating how model flexibility impacts
classifier decision boundaries. For this problem, please us
Scikit-learn's KNN implementation rather than your own implementation,
as you did at the end of the last question.

\textbf{5.1.} Create a synthetic dataset (with both features and
targets). Use the
\href{http://scikit-learn.org/stable/modules/generated/sklearn.datasets.make_moons.html\#sklearn.datasets.make_moons}{\texttt{make\_moons}}
module with the parameter \texttt{noise=0.35} to generate 1000 random
samples.

\textbf{5.2.} Visualize your data: scatterplot your random samples with
each class in a different color.

\textbf{5.3.} Create 3 different data subsets by selecting 100 of the
1000 data points at random three times (with replacement). For each of
these 100-sample datasets, fit three separate k-Nearest Neighbor
classifiers with: \(k = \{1, 25, 50\}\). This will result in 9
combinations (3 datasets, each with 3 trained classifiers).

\textbf{5.4.} For each combination of dataset and trained classifier
plot the decision boundary (similar in style to Figure 2.15 from
\emph{Introduction to Statistical Learning}). This should form a 3-by-3
grid. Each column should represent a different value of \(k\) and each
row should represent a different dataset.

\textbf{5.5.} What do you notice about the difference between the
decision boundaries in the rows and the columns in your figure? Which
decision boundaries appear to best separate the two classes of data with
respect to the training data? Which decision boundaries vary the most as
the training data change? Which decision boundaries do you anticipate
will generalize best to unseen data and why?

\textbf{5.6.} Explain the bias-variance tradeoff using the example of
the plots you made in this exercise and its implications for training
supervised machine learning algorithms.

Notes and tips for plotting decision boundaries (as in part 5.4): -
\emph{Resource for plotting decision boundaries with meshgrid and
contour:
https://scikit-learn.org/stable/auto\_examples/neighbors/plot\_classification.html}
- If you would like to change the colors of the background, and do not
like any of the existing cmap available in matplotlib, you can make your
own cmap using the 2 sets of rgb values. Sample code (replace r, g, b
with respective rgb values):

\begin{Shaded}
\begin{Highlighting}[]
\ImportTok{from}\NormalTok{ matplotlib.colors }\ImportTok{import}\NormalTok{ LinearSegmentedColormap}
\NormalTok{newcmp }\OperatorTok{=}\NormalTok{ LinearSegmentedColormap.from\_list(}\StringTok{"new"}\NormalTok{, [(r}\OperatorTok{/}\DecValTok{255}\NormalTok{, g}\OperatorTok{/}\DecValTok{255}\NormalTok{, b}\OperatorTok{/}\DecValTok{255}\NormalTok{), (r}\OperatorTok{/}\DecValTok{255}\NormalTok{, g}\OperatorTok{/}\DecValTok{255}\NormalTok{, b}\OperatorTok{/}\DecValTok{255}\NormalTok{)], N}\OperatorTok{=}\DecValTok{2}\NormalTok{)}
\end{Highlighting}
\end{Shaded}

\subsection{Exercise 6 - Bias-variance trade-off II: Quantifying the
tradeoff}\label{exercise-6---bias-variance-trade-off-ii-quantifying-the-tradeoff}

\textbf{{[}18 points{]}}

This exercise explores the impact of the bias-variance tradeoff on
classifier performance by looking at the performance on both training
and test data.

Here, the value of \(k\) determines how flexible our model is.

\textbf{6.1.} Using the function created earlier to generate random
samples (using the \texttt{make\_moons} function setting the
\texttt{noise} parameter to 0.35), create a new set of 1000 random
samples, and call this dataset your test set and the previously created
dataset your training set.

\textbf{6.2.} Train a kNN classifier on your training set for
\(k = 1,2,...500\). Apply each of these trained classifiers to both your
training dataset and your test dataset and plot the classification error
(fraction of incorrect predictions).

\textbf{6.3.} What trend do you see in the results?

\textbf{6.4.} What values of \(k\) represent high bias and which
represent high variance?

\textbf{6.5.} What is the optimal value of \(k\) and why?

\textbf{6.6.} In KNN classifiers, the value of k controls the
flexibility of the model - what controls the flexibility of other
models?

\subsection{Exercise 7 - Linear regression and nonlinear
transformations}\label{exercise-7---linear-regression-and-nonlinear-transformations}

\textbf{{[}18 points{]}}

\begin{tcolorbox}[enhanced jigsaw, opacityback=0, colframe=quarto-callout-note-color-frame, leftrule=.75mm, toptitle=1mm, colbacktitle=quarto-callout-note-color!10!white, bottomrule=.15mm, colback=white, coltitle=black, toprule=.15mm, opacitybacktitle=0.6, titlerule=0mm, arc=.35mm, bottomtitle=1mm, title=\textcolor{quarto-callout-note-color}{\faInfo}\hspace{0.5em}{Note}, rightrule=.15mm, breakable, left=2mm]

\href{https://github.com/kylebradbury/ids705/tree/main/notebooks/data/a1}{Data
for this exercise can be downloaded here}

\end{tcolorbox}

Linear regression can be used to model nonlinear relationships when
feature variables are properly transformed to represent the
nonlinearities in the data folder. In this exercise, you're given
training and test data contained in files ``A2\_Q7\_train.csv'' and
``A2\_Q7\_test.csv'' in the data. Your goal is to develop a regression
algorithm from the training data that performs well on the test data.

\emph{Hint: Use the scikit learn
\href{http://scikit-learn.org/stable/modules/generated/sklearn.linear_model.LinearRegression.html}{LinearRegression}
module.}

\textbf{7.1.} Create a scatter plot of your training data.

\textbf{7.2.} Estimate a linear regression model (\(y = a_0 + a_1 x\))
for the training data and calculate both the \(R^2\) value and mean
square error for the fit of that model for the training data. Also
provide the equation representing the estimated model
(e.g.~\(y = a_0 + a_1 x\), but with the estimated coefficients inserted.
Consider this your baseline model against which you will compare other
model options. \emph{Evaluating performance on the training data is not
a measure of how well this model would generalize to unseen data. We
will evaluate performance on the test data once we see our models fit
the training data decently well.}

\textbf{7.3.} If features can be nonlinearly transformed, a linear model
may incorporate those non-linear feature transformation relationships in
the training process. From looking at the scatter plot of the training
data, choose a transformation of the predictor variable, \(x\) that may
make sense for these data. This will be a multiple regression model of
the form \(y = a_0 + a_1 z_1 + a_2 z_2 + \ldots + a_n z_n\). Here
\(z_i\) could be any transformations of x - perhaps it's
\(\frac{1}{x}\), \(log(x)\), \(sin(x)\), \(x^k\) (where \(k\) is any
power of your choosing). Provide the estimated equation for this
multiple regression model (e.g.~if you chose your predictors to be
\(z_1 = x\) and \(z_2 = log(x)\), your model would be of the form
\(y = a_0 + a_1 x + a_2 log(x)\). Also provide the \(R^2\) and mean
square error of the fit for the training data.

\textbf{7.4.} Visualize the model fit to the training data. Using both
of the models you created in parts (b) and (c), plot the original data
(as a scatter plot) AND the curves representing your models (each as a
separate curve) from (b) and (c).

\textbf{7.5.} Now its time to compare your models and evaluate the
generalization performance on held out test data. Using the models above
from (b) an (c), apply them to the test data and estimate the \(R^2\)
and mean square error of the test dataset.

\textbf{7.6.} Which models perform better on the training data, and
which on the test data? Why?

\textbf{7.7.} Imagine that the test data were significantly different
from the training dataset. How might this affect the predictive
capability of your model? How would the accuracy of generalization
performance be impacted? Why?

\emph{To help get you started - here's some code to help you load in the
data for this exercise (you'll just need to update the path)}:

\begin{Shaded}
\begin{Highlighting}[]
\ImportTok{import}\NormalTok{ numpy }\ImportTok{as}\NormalTok{ np}
\ImportTok{import}\NormalTok{ pandas }\ImportTok{as}\NormalTok{ pd}

\NormalTok{path }\OperatorTok{=} \StringTok{\textquotesingle{}./data/a2\textquotesingle{}}
\NormalTok{train }\OperatorTok{=}\NormalTok{ pd.read\_csv(path }\OperatorTok{+} \StringTok{\textquotesingle{}A2\_Q7\_train.csv\textquotesingle{}}\NormalTok{)}
\NormalTok{test }\OperatorTok{=}\NormalTok{ pd.read\_csv(path }\OperatorTok{+} \StringTok{\textquotesingle{}A2\_Q7\_test.csv\textquotesingle{}}\NormalTok{)}

\NormalTok{x\_train }\OperatorTok{=}\NormalTok{ train.x.values}
\NormalTok{y\_train }\OperatorTok{=}\NormalTok{ train.y.values}

\NormalTok{x\_test }\OperatorTok{=}\NormalTok{ test.x.values}
\NormalTok{y\_test }\OperatorTok{=}\NormalTok{ test.y.values}
\end{Highlighting}
\end{Shaded}





\end{document}
