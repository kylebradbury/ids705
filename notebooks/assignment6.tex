% Options for packages loaded elsewhere
\PassOptionsToPackage{unicode}{hyperref}
\PassOptionsToPackage{hyphens}{url}
\PassOptionsToPackage{dvipsnames,svgnames,x11names}{xcolor}
%
\documentclass[
  letterpaper,
  DIV=11,
  numbers=noendperiod]{scrartcl}

\usepackage{amsmath,amssymb}
\usepackage{iftex}
\ifPDFTeX
  \usepackage[T1]{fontenc}
  \usepackage[utf8]{inputenc}
  \usepackage{textcomp} % provide euro and other symbols
\else % if luatex or xetex
  \usepackage{unicode-math}
  \defaultfontfeatures{Scale=MatchLowercase}
  \defaultfontfeatures[\rmfamily]{Ligatures=TeX,Scale=1}
\fi
\usepackage{lmodern}
\ifPDFTeX\else  
    % xetex/luatex font selection
\fi
% Use upquote if available, for straight quotes in verbatim environments
\IfFileExists{upquote.sty}{\usepackage{upquote}}{}
\IfFileExists{microtype.sty}{% use microtype if available
  \usepackage[]{microtype}
  \UseMicrotypeSet[protrusion]{basicmath} % disable protrusion for tt fonts
}{}
\makeatletter
\@ifundefined{KOMAClassName}{% if non-KOMA class
  \IfFileExists{parskip.sty}{%
    \usepackage{parskip}
  }{% else
    \setlength{\parindent}{0pt}
    \setlength{\parskip}{6pt plus 2pt minus 1pt}}
}{% if KOMA class
  \KOMAoptions{parskip=half}}
\makeatother
\usepackage{xcolor}
\setlength{\emergencystretch}{3em} % prevent overfull lines
\setcounter{secnumdepth}{-\maxdimen} % remove section numbering
% Make \paragraph and \subparagraph free-standing
\makeatletter
\ifx\paragraph\undefined\else
  \let\oldparagraph\paragraph
  \renewcommand{\paragraph}{
    \@ifstar
      \xxxParagraphStar
      \xxxParagraphNoStar
  }
  \newcommand{\xxxParagraphStar}[1]{\oldparagraph*{#1}\mbox{}}
  \newcommand{\xxxParagraphNoStar}[1]{\oldparagraph{#1}\mbox{}}
\fi
\ifx\subparagraph\undefined\else
  \let\oldsubparagraph\subparagraph
  \renewcommand{\subparagraph}{
    \@ifstar
      \xxxSubParagraphStar
      \xxxSubParagraphNoStar
  }
  \newcommand{\xxxSubParagraphStar}[1]{\oldsubparagraph*{#1}\mbox{}}
  \newcommand{\xxxSubParagraphNoStar}[1]{\oldsubparagraph{#1}\mbox{}}
\fi
\makeatother

\usepackage{color}
\usepackage{fancyvrb}
\newcommand{\VerbBar}{|}
\newcommand{\VERB}{\Verb[commandchars=\\\{\}]}
\DefineVerbatimEnvironment{Highlighting}{Verbatim}{commandchars=\\\{\}}
% Add ',fontsize=\small' for more characters per line
\usepackage{framed}
\definecolor{shadecolor}{RGB}{241,243,245}
\newenvironment{Shaded}{\begin{snugshade}}{\end{snugshade}}
\newcommand{\AlertTok}[1]{\textcolor[rgb]{0.68,0.00,0.00}{#1}}
\newcommand{\AnnotationTok}[1]{\textcolor[rgb]{0.37,0.37,0.37}{#1}}
\newcommand{\AttributeTok}[1]{\textcolor[rgb]{0.40,0.45,0.13}{#1}}
\newcommand{\BaseNTok}[1]{\textcolor[rgb]{0.68,0.00,0.00}{#1}}
\newcommand{\BuiltInTok}[1]{\textcolor[rgb]{0.00,0.23,0.31}{#1}}
\newcommand{\CharTok}[1]{\textcolor[rgb]{0.13,0.47,0.30}{#1}}
\newcommand{\CommentTok}[1]{\textcolor[rgb]{0.37,0.37,0.37}{#1}}
\newcommand{\CommentVarTok}[1]{\textcolor[rgb]{0.37,0.37,0.37}{\textit{#1}}}
\newcommand{\ConstantTok}[1]{\textcolor[rgb]{0.56,0.35,0.01}{#1}}
\newcommand{\ControlFlowTok}[1]{\textcolor[rgb]{0.00,0.23,0.31}{\textbf{#1}}}
\newcommand{\DataTypeTok}[1]{\textcolor[rgb]{0.68,0.00,0.00}{#1}}
\newcommand{\DecValTok}[1]{\textcolor[rgb]{0.68,0.00,0.00}{#1}}
\newcommand{\DocumentationTok}[1]{\textcolor[rgb]{0.37,0.37,0.37}{\textit{#1}}}
\newcommand{\ErrorTok}[1]{\textcolor[rgb]{0.68,0.00,0.00}{#1}}
\newcommand{\ExtensionTok}[1]{\textcolor[rgb]{0.00,0.23,0.31}{#1}}
\newcommand{\FloatTok}[1]{\textcolor[rgb]{0.68,0.00,0.00}{#1}}
\newcommand{\FunctionTok}[1]{\textcolor[rgb]{0.28,0.35,0.67}{#1}}
\newcommand{\ImportTok}[1]{\textcolor[rgb]{0.00,0.46,0.62}{#1}}
\newcommand{\InformationTok}[1]{\textcolor[rgb]{0.37,0.37,0.37}{#1}}
\newcommand{\KeywordTok}[1]{\textcolor[rgb]{0.00,0.23,0.31}{\textbf{#1}}}
\newcommand{\NormalTok}[1]{\textcolor[rgb]{0.00,0.23,0.31}{#1}}
\newcommand{\OperatorTok}[1]{\textcolor[rgb]{0.37,0.37,0.37}{#1}}
\newcommand{\OtherTok}[1]{\textcolor[rgb]{0.00,0.23,0.31}{#1}}
\newcommand{\PreprocessorTok}[1]{\textcolor[rgb]{0.68,0.00,0.00}{#1}}
\newcommand{\RegionMarkerTok}[1]{\textcolor[rgb]{0.00,0.23,0.31}{#1}}
\newcommand{\SpecialCharTok}[1]{\textcolor[rgb]{0.37,0.37,0.37}{#1}}
\newcommand{\SpecialStringTok}[1]{\textcolor[rgb]{0.13,0.47,0.30}{#1}}
\newcommand{\StringTok}[1]{\textcolor[rgb]{0.13,0.47,0.30}{#1}}
\newcommand{\VariableTok}[1]{\textcolor[rgb]{0.07,0.07,0.07}{#1}}
\newcommand{\VerbatimStringTok}[1]{\textcolor[rgb]{0.13,0.47,0.30}{#1}}
\newcommand{\WarningTok}[1]{\textcolor[rgb]{0.37,0.37,0.37}{\textit{#1}}}

\providecommand{\tightlist}{%
  \setlength{\itemsep}{0pt}\setlength{\parskip}{0pt}}\usepackage{longtable,booktabs,array}
\usepackage{calc} % for calculating minipage widths
% Correct order of tables after \paragraph or \subparagraph
\usepackage{etoolbox}
\makeatletter
\patchcmd\longtable{\par}{\if@noskipsec\mbox{}\fi\par}{}{}
\makeatother
% Allow footnotes in longtable head/foot
\IfFileExists{footnotehyper.sty}{\usepackage{footnotehyper}}{\usepackage{footnote}}
\makesavenoteenv{longtable}
\usepackage{graphicx}
\makeatletter
\newsavebox\pandoc@box
\newcommand*\pandocbounded[1]{% scales image to fit in text height/width
  \sbox\pandoc@box{#1}%
  \Gscale@div\@tempa{\textheight}{\dimexpr\ht\pandoc@box+\dp\pandoc@box\relax}%
  \Gscale@div\@tempb{\linewidth}{\wd\pandoc@box}%
  \ifdim\@tempb\p@<\@tempa\p@\let\@tempa\@tempb\fi% select the smaller of both
  \ifdim\@tempa\p@<\p@\scalebox{\@tempa}{\usebox\pandoc@box}%
  \else\usebox{\pandoc@box}%
  \fi%
}
% Set default figure placement to htbp
\def\fps@figure{htbp}
\makeatother

\KOMAoption{captions}{tableheading}
\makeatletter
\@ifpackageloaded{caption}{}{\usepackage{caption}}
\AtBeginDocument{%
\ifdefined\contentsname
  \renewcommand*\contentsname{Table of contents}
\else
  \newcommand\contentsname{Table of contents}
\fi
\ifdefined\listfigurename
  \renewcommand*\listfigurename{List of Figures}
\else
  \newcommand\listfigurename{List of Figures}
\fi
\ifdefined\listtablename
  \renewcommand*\listtablename{List of Tables}
\else
  \newcommand\listtablename{List of Tables}
\fi
\ifdefined\figurename
  \renewcommand*\figurename{Figure}
\else
  \newcommand\figurename{Figure}
\fi
\ifdefined\tablename
  \renewcommand*\tablename{Table}
\else
  \newcommand\tablename{Table}
\fi
}
\@ifpackageloaded{float}{}{\usepackage{float}}
\floatstyle{ruled}
\@ifundefined{c@chapter}{\newfloat{codelisting}{h}{lop}}{\newfloat{codelisting}{h}{lop}[chapter]}
\floatname{codelisting}{Listing}
\newcommand*\listoflistings{\listof{codelisting}{List of Listings}}
\makeatother
\makeatletter
\makeatother
\makeatletter
\@ifpackageloaded{caption}{}{\usepackage{caption}}
\@ifpackageloaded{subcaption}{}{\usepackage{subcaption}}
\makeatother

\usepackage{bookmark}

\IfFileExists{xurl.sty}{\usepackage{xurl}}{} % add URL line breaks if available
\urlstyle{same} % disable monospaced font for URLs
\hypersetup{
  pdftitle={Assignment 6},
  pdfauthor={Kyle Bradbury},
  colorlinks=true,
  linkcolor={blue},
  filecolor={Maroon},
  citecolor={Blue},
  urlcolor={Blue},
  pdfcreator={LaTeX via pandoc}}


\title{Assignment 6}
\usepackage{etoolbox}
\makeatletter
\providecommand{\subtitle}[1]{% add subtitle to \maketitle
  \apptocmd{\@title}{\par {\large #1 \par}}{}{}
}
\makeatother
\subtitle{Reinforcement Learning}
\author{Kyle Bradbury}
\date{2025-03-27}

\begin{document}
\maketitle

\renewcommand*\contentsname{Table of contents}
{
\hypersetup{linkcolor=}
\setcounter{tocdepth}{1}
\tableofcontents
}

\subsection{Instructions}\label{instructions}

\emph{Instructions for all assignments can be found
\href{https://kylebradbury.github.io/ids705/notebooks/assignment_instructions.html}{here}.
Note: this assignment falls under collaboration Mode 2: Individual
Assignment -- Collaboration Permitted. Please refer to the syllabus for
additional information. Please be sure to list the names of any students
that you worked with on this assignment. Total points in the assignment
add up to 90; an additional 10 points are allocated to professionalism
and presentation quality.}

\subsection{Learning objectives}\label{learning-objectives}

After completing this assignment, you will be able to\ldots{} - Clearly
articulate the role of the key components of reinforcement learning: the
agent, actions, rewards, policies, state values, and action values. -
Apply policy evaluation to a problem in practice - Use Monte Carlo
control to determine and apply an optimal policy for a reinforcement
learning problem and learn an optimal strategy from trial and error,
alone

\subsection{Background on our Blackjack
variant}\label{background-on-our-blackjack-variant}

Your goal is to develop a reinforcement learning technique to learn the
optimal policy for winning at blackjack through trial-and-error
learning. Here, we're going to modify the rules from traditional
blackjack a bit in a way that corresponds to the game presented in
Sutton and Barto's \emph{Reinforcement Learning: An Introduction}
(Chapter 5, example 5.1). A full implementation of the game is provided
and usage examples are detailed in the class header below.

The rules of this modified version of the game of blackjack are as
follows:

\begin{itemize}
\tightlist
\item
  Blackjack is a card game where the goal is to obtain cards that sum to
  as near as possible to 21 without going over. We're playing against a
  fixed (autonomous) dealer.
\item
  Face cards (Jack, Queen, King) have point value 10. Aces can either
  count as 11 or 1 (whichever is most advantageous to the player), and
  we're refer to it as a `usable' Ace if we're treating it as 11
  (indicating that it could be used as a `1', instead, if need be). This
  game is played with a deck of cards sampled with replacement.
\item
  The game starts with both the player and the dealer having one face up
  and one face down card.
\item
  The player can request additional cards (known as taking a ``hit'',
  which we define as action ``1'') until either they decide to stop
  (known as ``staying'', which we define as action `0') or their cards
  exceed 21 (known as a ``bust'', at which time the game ends and player
  loses).
\item
  If the player stays (and hasn't exceeded a score of 21), the dealer
  reveals their facedown card, and draws until their sum is 17 or
  greater. If the dealer busts the player wins. If neither player nor
  dealer busts, the outcome (win, lose, draw) is decided by whose sum is
  closer to 21. The reward for winning is +1, tying is 0, and losing is
  -1.
\end{itemize}

Over the course of these exercises, you will accomplish three things:

\begin{enumerate}
\def\labelenumi{\arabic{enumi}.}
\tightlist
\item
  Try your hand at this game of blackjack and see what your human
  reinforcement learning system is able to achieve
\item
  Evaluate a naive policy using Monte Carlo policy evaluation
\item
  Determine an optimal policy using Monte Carlo control
\end{enumerate}

\emph{This problem is adapted from David Silver's
\href{http://www0.cs.ucl.ac.uk/staff/d.silver/web/Teaching.html}{excellent
series on Reinforcement Learning} at University College London}

\subsection{Exercise 1 - Human reinforcement
learning}\label{exercise-1---human-reinforcement-learning}

\textbf{{[}5 points{]}}

\textbf{1.1.} Using the code detailed below, play at least 20 hands of
the modified blackjack game below, and record your returns (average
cumulative reward) across all episodes. This will help you get
accustomed with how the game works, the data structures involved with
representing states, and what strategies are most effective. Since for
this game, you only get a nonzero reward at the end of the episode, you
can simply play the game and note the reward in a spreadsheet, then
average across all plays of the game (of course, you're welcome to code
up a way to do that automatically if you're so inspired).

\begin{Shaded}
\begin{Highlighting}[]
\ImportTok{import}\NormalTok{ numpy }\ImportTok{as}\NormalTok{ np}

\KeywordTok{class}\NormalTok{ Blackjack():}
    \CommentTok{"""Simple blackjack environment adapted from OpenAI Gym:}
\CommentTok{        https://github.com/openai/gym/blob/master/gym/envs/toy\_text/blackjack.py}
\CommentTok{    }
\CommentTok{    Blackjack is a card game where the goal is to obtain cards that sum to as}
\CommentTok{    near as possible to 21 without going over.  They\textquotesingle{}re playing against a fixed}
\CommentTok{    dealer.}
\CommentTok{    }
\CommentTok{    Face cards (Jack, Queen, King) have point value 10.}
\CommentTok{    Aces can either count as 11 or 1, and it\textquotesingle{}s called \textquotesingle{}usable\textquotesingle{} at 11.}
\CommentTok{    This game is placed with a deck sampled with replacement.}
\CommentTok{    }
\CommentTok{    The game starts with each (player and dealer) having one face up and one}
\CommentTok{    face down card.}
\CommentTok{    }
\CommentTok{    The player can request additional cards (hit = 1) until they decide to stop}
\CommentTok{    (stay = 0) or exceed 21 (bust).}
\CommentTok{    }
\CommentTok{    After the player stays, the dealer reveals their facedown card, and draws}
\CommentTok{    until their sum is 17 or greater.  If the dealer goes bust the player wins.}
\CommentTok{    If neither player nor dealer busts, the outcome (win, lose, draw) is}
\CommentTok{    decided by whose sum is closer to 21.  The reward for winning is +1,}
\CommentTok{    drawing is 0, and losing is {-}1.}
\CommentTok{    }
\CommentTok{    The observation is a 3{-}tuple of: the players current sum,}
\CommentTok{    the dealer\textquotesingle{}s one showing card (1{-}10 where 1 is ace),}
\CommentTok{    and whether or not the player holds a usable ace (0 or 1).}
\CommentTok{    }
\CommentTok{    This environment corresponds to the version of the blackjack problem}
\CommentTok{    described in Example 5.1 in Reinforcement Learning: An Introduction}
\CommentTok{    by Sutton and Barto (1998).}
\CommentTok{    }
\CommentTok{    http://incompleteideas.net/sutton/book/the{-}book.html}
\CommentTok{    }
\CommentTok{    Usage: }
\CommentTok{        Initialize the class:}
\CommentTok{            game = Blackjack()}
\CommentTok{        }
\CommentTok{        Deal the cards:}
\CommentTok{            game.deal()}
\CommentTok{            }
\CommentTok{             (14, 3, False)}
\CommentTok{             }
\CommentTok{            This is the agent\textquotesingle{}s observation of the state of the game:}
\CommentTok{            The first value is the sum of cards in your hand (14 in this case)}
\CommentTok{            The second is the visible card in the dealer\textquotesingle{}s hand (3 in this case)}
\CommentTok{            The Boolean is a flag (False in this case) to indicate whether or }
\CommentTok{                not you have a usable Ace}
\CommentTok{            (Note: if you have a usable ace, the sum will treat the ace as a }
\CommentTok{                value of \textquotesingle{}11\textquotesingle{} {-} this is the case if this Boolean flag is "true")}
\CommentTok{            }
\CommentTok{        Take an action: Hit (1) or stay (0)}
\CommentTok{        }
\CommentTok{            Take a hit: game.step(1)}
\CommentTok{            To Stay:    game.step(0)}
\CommentTok{            }
\CommentTok{        The output summarizes the game status:}
\CommentTok{            }
\CommentTok{            ((15, 3, False), 0, False)}
\CommentTok{            }
\CommentTok{            The first tuple (15, 3, False), is the agent\textquotesingle{}s observation of the}
\CommentTok{            state of the game as described above.}
\CommentTok{            The second value (0) indicates the rewards}
\CommentTok{            The third value (False) indicates whether the game is finished}
\CommentTok{    """}
    
    \KeywordTok{def} \FunctionTok{\_\_init\_\_}\NormalTok{(}\VariableTok{self}\NormalTok{):}
        \CommentTok{\# 1 = Ace, 2{-}10 = Number cards, Jack/Queen/King = 10}
        \VariableTok{self}\NormalTok{.deck   }\OperatorTok{=}\NormalTok{ [}\DecValTok{1}\NormalTok{, }\DecValTok{2}\NormalTok{, }\DecValTok{3}\NormalTok{, }\DecValTok{4}\NormalTok{, }\DecValTok{5}\NormalTok{, }\DecValTok{6}\NormalTok{, }\DecValTok{7}\NormalTok{, }\DecValTok{8}\NormalTok{, }\DecValTok{9}\NormalTok{, }\DecValTok{10}\NormalTok{, }\DecValTok{10}\NormalTok{, }\DecValTok{10}\NormalTok{, }\DecValTok{10}\NormalTok{]}
        \VariableTok{self}\NormalTok{.dealer }\OperatorTok{=}\NormalTok{ []}
        \VariableTok{self}\NormalTok{.player }\OperatorTok{=}\NormalTok{ []}
        \VariableTok{self}\NormalTok{.deal()}

    \KeywordTok{def}\NormalTok{ step(}\VariableTok{self}\NormalTok{, action):}
        \ControlFlowTok{if}\NormalTok{ action }\OperatorTok{==} \DecValTok{1}\NormalTok{:  }\CommentTok{\# hit: add a card to players hand and return}
            \VariableTok{self}\NormalTok{.player.append(}\VariableTok{self}\NormalTok{.draw\_card())}
            \ControlFlowTok{if} \VariableTok{self}\NormalTok{.is\_bust(}\VariableTok{self}\NormalTok{.player):}
\NormalTok{                done }\OperatorTok{=} \VariableTok{True}
\NormalTok{                reward }\OperatorTok{=} \OperatorTok{{-}}\DecValTok{1}
            \ControlFlowTok{else}\NormalTok{:}
\NormalTok{                done }\OperatorTok{=} \VariableTok{False}
\NormalTok{                reward }\OperatorTok{=} \DecValTok{0}
        \ControlFlowTok{else}\NormalTok{:  }\CommentTok{\# stay: play out the dealers hand, and score}
\NormalTok{            done }\OperatorTok{=} \VariableTok{True}
            \ControlFlowTok{while} \VariableTok{self}\NormalTok{.sum\_hand(}\VariableTok{self}\NormalTok{.dealer) }\OperatorTok{\textless{}} \DecValTok{17}\NormalTok{:}
                \VariableTok{self}\NormalTok{.dealer.append(}\VariableTok{self}\NormalTok{.draw\_card())}
\NormalTok{            reward }\OperatorTok{=} \VariableTok{self}\NormalTok{.}\BuiltInTok{cmp}\NormalTok{(}\VariableTok{self}\NormalTok{.score(}\VariableTok{self}\NormalTok{.player), }\VariableTok{self}\NormalTok{.score(}\VariableTok{self}\NormalTok{.dealer))}
        \ControlFlowTok{return} \VariableTok{self}\NormalTok{.\_get\_obs(), reward, done}

    \KeywordTok{def}\NormalTok{ \_get\_obs(}\VariableTok{self}\NormalTok{):}
        \ControlFlowTok{return}\NormalTok{ (}\VariableTok{self}\NormalTok{.sum\_hand(}\VariableTok{self}\NormalTok{.player), }\VariableTok{self}\NormalTok{.dealer[}\DecValTok{0}\NormalTok{], }\VariableTok{self}\NormalTok{.usable\_ace(}\VariableTok{self}\NormalTok{.player))}

    \KeywordTok{def}\NormalTok{ deal(}\VariableTok{self}\NormalTok{):}
        \VariableTok{self}\NormalTok{.dealer }\OperatorTok{=} \VariableTok{self}\NormalTok{.draw\_hand()}
        \VariableTok{self}\NormalTok{.player }\OperatorTok{=} \VariableTok{self}\NormalTok{.draw\_hand()}
        \ControlFlowTok{return} \VariableTok{self}\NormalTok{.\_get\_obs()}
    
    \CommentTok{\#{-}{-}{-}{-}{-}{-}{-}{-}{-}{-}{-}{-}{-}{-}{-}{-}{-}{-}{-}{-}{-}{-}{-}{-}{-}{-}{-}{-}{-}{-}{-}{-}{-}{-}{-}{-}{-}{-}{-}{-}{-}{-}}
    \CommentTok{\# Other helper functions}
    \CommentTok{\#{-}{-}{-}{-}{-}{-}{-}{-}{-}{-}{-}{-}{-}{-}{-}{-}{-}{-}{-}{-}{-}{-}{-}{-}{-}{-}{-}{-}{-}{-}{-}{-}{-}{-}{-}{-}{-}{-}{-}{-}{-}{-}}
    \KeywordTok{def} \BuiltInTok{cmp}\NormalTok{(}\VariableTok{self}\NormalTok{, a, b):}
        \ControlFlowTok{return} \BuiltInTok{float}\NormalTok{(a }\OperatorTok{\textgreater{}}\NormalTok{ b) }\OperatorTok{{-}} \BuiltInTok{float}\NormalTok{(a }\OperatorTok{\textless{}}\NormalTok{ b)}
    
    \KeywordTok{def}\NormalTok{ draw\_card(}\VariableTok{self}\NormalTok{):}
        \ControlFlowTok{return} \BuiltInTok{int}\NormalTok{(np.random.choice(}\VariableTok{self}\NormalTok{.deck))}
    
    \KeywordTok{def}\NormalTok{ draw\_hand(}\VariableTok{self}\NormalTok{):}
        \ControlFlowTok{return}\NormalTok{ [}\VariableTok{self}\NormalTok{.draw\_card(), }\VariableTok{self}\NormalTok{.draw\_card()]}
    
    \KeywordTok{def}\NormalTok{ usable\_ace(}\VariableTok{self}\NormalTok{,hand):  }\CommentTok{\# Does this hand have a usable ace?}
        \ControlFlowTok{return} \DecValTok{1} \KeywordTok{in}\NormalTok{ hand }\KeywordTok{and} \BuiltInTok{sum}\NormalTok{(hand) }\OperatorTok{+} \DecValTok{10} \OperatorTok{\textless{}=} \DecValTok{21}
    
    \KeywordTok{def}\NormalTok{ sum\_hand(}\VariableTok{self}\NormalTok{,hand):  }\CommentTok{\# Return current hand total}
        \ControlFlowTok{if} \VariableTok{self}\NormalTok{.usable\_ace(hand):}
            \ControlFlowTok{return} \BuiltInTok{sum}\NormalTok{(hand) }\OperatorTok{+} \DecValTok{10}
        \ControlFlowTok{return} \BuiltInTok{sum}\NormalTok{(hand)}
    
    \KeywordTok{def}\NormalTok{ is\_bust(}\VariableTok{self}\NormalTok{,hand):  }\CommentTok{\# Is this hand a bust?}
        \ControlFlowTok{return} \VariableTok{self}\NormalTok{.sum\_hand(hand) }\OperatorTok{\textgreater{}} \DecValTok{21}
    
    \KeywordTok{def}\NormalTok{ score(}\VariableTok{self}\NormalTok{,hand):  }\CommentTok{\# What is the score of this hand (0 if bust)}
        \ControlFlowTok{return} \DecValTok{0} \ControlFlowTok{if} \VariableTok{self}\NormalTok{.is\_bust(hand) }\ControlFlowTok{else} \VariableTok{self}\NormalTok{.sum\_hand(hand)}
\end{Highlighting}
\end{Shaded}

Here's an example of how it works to get you started:

\begin{Shaded}
\begin{Highlighting}[]
\ImportTok{import}\NormalTok{ numpy }\ImportTok{as}\NormalTok{ np}

\CommentTok{\# Initialize the class:}
\NormalTok{game }\OperatorTok{=}\NormalTok{ Blackjack()}

\CommentTok{\# Deal the cards:}
\NormalTok{s0 }\OperatorTok{=}\NormalTok{ game.deal()}
\BuiltInTok{print}\NormalTok{(s0)}

\CommentTok{\# Take an action: Hit = 1 or stay = 0. Here\textquotesingle{}s a hit:}
\NormalTok{s1 }\OperatorTok{=}\NormalTok{ game.step(}\DecValTok{1}\NormalTok{)}
\BuiltInTok{print}\NormalTok{(s1)}

\CommentTok{\# If you wanted to stay:}
\CommentTok{\# game.step(2)}

\CommentTok{\# When it\textquotesingle{}s gameover, just redeal:}
\CommentTok{\# game.deal()}
\end{Highlighting}
\end{Shaded}

\begin{verbatim}
(12, 10, False)
((22, 10, False), -1, True)
\end{verbatim}

\textbf{Sample Anwer}

My rewards for each episode played were: {[}??, ??, ??, ??,
\ldots\ldots{]}

My average reward was {[}INSERT HERE{]} after playing ?? hands. {[}Note
- you must play at least 20 hands{]}

\subsection{Exercise 2 - Perform Monte Carlo Policy
Evaluation}\label{exercise-2---perform-monte-carlo-policy-evaluation}

\textbf{{[}40 points{]}}

Thinking that you want to make your millions playing this modified
version of blackjack, you decide to test out a policy for playing this
game. Your idea is an aggressive strategy: always hit unless the total
of your cards adds up to 20 or 21, in which case you stay.

\textbf{2.1.} Use Monte Carlo policy evaluation to evaluate the expected
returns from each state. Create plots for these similar to Sutton and
Barto, Figure 5.1 where you plot the expected returns for each state. In
this case create 2 plots (sample code is provided below):

\begin{enumerate}
\def\labelenumi{\arabic{enumi}.}
\tightlist
\item
  When you have a usable ace, plot the value function showing the
  dealer's card on the x-axis, and the player's sum on the y-axis, and
  use a colormap to plot the state value corresponding to each state
  under the policy described above. The domain of your x and y axes
  should include all possible states (2 to 21 for the player sum, and 1
  to 10 for the dealer's card). Show the estimated state value function
  after 10,000 episodes.
\item
  Repeat (1) for the states without a usable ace (code is also provided
  for this).
\item
  Repeat (1) after 500,000 episodes.
\item
  Repeat (2) after 500,000 episodes.
\end{enumerate}

\textbf{2.2.} Show a plot of the cumulative average reward (returns) per
episode vs the number of episodes.

\begin{itemize}
\tightlist
\item
  For this plot, make the x-axis log scale
\item
  For both the 10,000 episode case and the 500,000 episode case, state
  the final value of that average returns for this policy in those two
  cases (these are just the values of the plot of cumulative average
  reward at iteration 10,000 and 500,000, respectively), you can write
  this in a line of text.
\end{itemize}

\emph{Note on sample code: the code provided for these questions is
meant to be a helpful starting point - you are not required to fill it
out exactly or use all components, but it is meant to help you as you
begin thinking about this problem.}

\textbf{Starter Code}

\begin{Shaded}
\begin{Highlighting}[]
\KeywordTok{def}\NormalTok{ policy\_evaluation(N\_episodes):}
    \CommentTok{\# Initialize variables}
\NormalTok{    avg\_return }\OperatorTok{=}\NormalTok{ []    }\CommentTok{\# Accumulator to store the average returns over all }
                       \CommentTok{\#  episodes}
\NormalTok{    episode\_count }\OperatorTok{=} \DecValTok{0}  \CommentTok{\# Number of episodes played (will increment as more }
                       \CommentTok{\#  episodes are played)}
\NormalTok{    N\_player   }\OperatorTok{=} \DecValTok{32}    \CommentTok{\# Number of possible states of the players hand}
                       \CommentTok{\#  The max hand value is 21, but then if the player busts }
                       \CommentTok{\#  they could reach 32 (if they had 21 and took a hit)}
\NormalTok{    N\_dealer   }\OperatorTok{=} \DecValTok{10}    \CommentTok{\# Maximum number of states the dealer\textquotesingle{}s hand could take on}
                       \CommentTok{\#  1 through 10 (Ace could be 1 or 11, but would still }
                       \CommentTok{\#  be a singe card in hand)}
\NormalTok{    N\_ace      }\OperatorTok{=} \DecValTok{2}     \CommentTok{\# There are two states for the ace: }
                       \CommentTok{\#  (1) No usable ace, (2) Usable ace}
    
    \CommentTok{\# Initialize the state value function (in this case, the value function is  }
    \CommentTok{\#   the average return over episodes, so no need for an explicit returns }
    \CommentTok{\#   array)}
\NormalTok{    v }\OperatorTok{=}\NormalTok{ np.zeros((N\_player, N\_dealer, N\_ace)) }
    
    \CommentTok{\# Initialize a variable counting the number of visits to each state}
\NormalTok{    n\_visits }\OperatorTok{=}\NormalTok{ np.zeros((N\_player, N\_dealer, N\_ace))}
    
    \CommentTok{\# Initialize the policy that stays only if the player has 20 or 21, }
    \CommentTok{\#  otherwise hit}
    \CommentTok{\#\#\#\#\#\#\#\#\#\#\#\#\#\#\#\#\#\#\#\#\#\#\#}
    \CommentTok{\# FILL IN THIS CODE}
    \CommentTok{\#\#\#\#\#\#\#\#\#\#\#\#\#\#\#\#\#\#\#\#\#\#\#}
    
    \CommentTok{\# Load the game}
\NormalTok{    B }\OperatorTok{=}\NormalTok{ Blackjack()}
    
    \CommentTok{\# HELPER FUNCTIONS}
    \CommentTok{\# Convert the current state into a set of indices for the value function}
    \KeywordTok{def}\NormalTok{ state\_to\_index(s):}
\NormalTok{        ace }\OperatorTok{=} \DecValTok{0}
        \ControlFlowTok{if}\NormalTok{ s[}\DecValTok{2}\NormalTok{]:}
\NormalTok{            ace }\OperatorTok{=} \DecValTok{1}
\NormalTok{        index }\OperatorTok{=}\NormalTok{ [s[}\DecValTok{0}\NormalTok{]}\OperatorTok{{-}}\DecValTok{1}\NormalTok{, s[}\DecValTok{1}\NormalTok{]}\OperatorTok{{-}}\DecValTok{1}\NormalTok{, ace]}
        \ControlFlowTok{return}\NormalTok{ index}
    
    \CommentTok{\# Choose an action based on the policy, pi, and the current state, s}
    \KeywordTok{def}\NormalTok{ choose\_action(pi,s):}
        \CommentTok{\#\#\#\#\#\#\#\#\#\#\#\#\#\#\#\#\#\#\#\#\#\#\#}
        \CommentTok{\# FILL IN THIS FUNCTION}
        \CommentTok{\#\#\#\#\#\#\#\#\#\#\#\#\#\#\#\#\#\#\#\#\#\#\#}
        \CommentTok{\# Outputs a value of 0 or 1}
    
    \CommentTok{\# {-}{-}{-}{-}{-}{-}{-}{-}{-}{-}{-}{-}{-}{-}{-}{-}{-}{-}{-}{-}{-}{-}{-}{-}{-}{-}{-}{-}{-}{-}{-}{-}{-}{-}{-}{-}{-}{-}{-}{-}{-}{-}{-}{-}{-}{-}{-}{-}{-}{-}{-}{-}{-}{-}{-}{-}{-}{-}{-}{-}{-}{-}{-}{-}{-}{-}{-}{-}{-}{-}}
    \CommentTok{\# Run the policy evaluation   }
    \CommentTok{\# {-}{-}{-}{-}{-}{-}{-}{-}{-}{-}{-}{-}{-}{-}{-}{-}{-}{-}{-}{-}{-}{-}{-}{-}{-}{-}{-}{-}{-}{-}{-}{-}{-}{-}{-}{-}{-}{-}{-}{-}{-}{-}{-}{-}{-}{-}{-}{-}{-}{-}{-}{-}{-}{-}{-}{-}{-}{-}{-}{-}{-}{-}{-}{-}{-}{-}{-}{-}{-}{-}}
    \ControlFlowTok{for}\NormalTok{ episode }\KeywordTok{in} \BuiltInTok{range}\NormalTok{(N\_episodes):}
           
        \CommentTok{\# Deal a hand of blackjack}
    
        \CommentTok{\# Initialize a variable to hold the list of states visited and add to }
        \CommentTok{\#  the list the initial state}
        
        \CommentTok{\# Play the hand through, following the policy}
        
        \CommentTok{\# Update the average returns}

        \CommentTok{\# Update your state value function}

    \ControlFlowTok{return}\NormalTok{ (v, avg\_return)}
\end{Highlighting}
\end{Shaded}

\begin{Shaded}
\begin{Highlighting}[]
\CommentTok{\# {-}{-}{-}{-}{-}{-}{-}{-}{-}{-}{-}{-}{-}{-}{-}{-}{-}{-}{-}{-}{-}{-}{-}{-}{-}{-}{-}{-}{-}{-}{-}{-}{-}{-}{-}{-}{-}{-}{-}{-}{-}{-}{-}{-}{-}{-}{-}{-}{-}{-}{-}{-}{-}{-}{-}{-}{-}{-}{-}{-}{-}{-}{-}{-}{-}{-}{-}{-}{-}{-}{-}{-}{-}{-}{-}}
\CommentTok{\# Sample plotting function for the state value function}
\CommentTok{\# {-}{-}{-}{-}{-}{-}{-}{-}{-}{-}{-}{-}{-}{-}{-}{-}{-}{-}{-}{-}{-}{-}{-}{-}{-}{-}{-}{-}{-}{-}{-}{-}{-}{-}{-}{-}{-}{-}{-}{-}{-}{-}{-}{-}{-}{-}{-}{-}{-}{-}{-}{-}{-}{-}{-}{-}{-}{-}{-}{-}{-}{-}{-}{-}{-}{-}{-}{-}{-}{-}{-}{-}{-}{-}{-}        }

\ImportTok{import}\NormalTok{ matplotlib.pyplot }\ImportTok{as}\NormalTok{ plt}
\CommentTok{\# Uncomment the line below on macs for clearer plots:}
\CommentTok{\# \%config InlineBackend.figure\_format = \textquotesingle{}retina\textquotesingle{} }

\CommentTok{\# Plot the resulting state value function (expected returns from each state)}
\KeywordTok{def}\NormalTok{ plot\_value(v):}
\NormalTok{    plt.figure(figsize}\OperatorTok{=}\NormalTok{(}\DecValTok{10}\NormalTok{,}\DecValTok{7}\NormalTok{))}
\NormalTok{    drange }\OperatorTok{=}\NormalTok{ [}\DecValTok{1}\NormalTok{,}\DecValTok{10}\NormalTok{] }\CommentTok{\# Dealer range}
\NormalTok{    prange }\OperatorTok{=}\NormalTok{ [}\DecValTok{4}\NormalTok{,}\DecValTok{21}\NormalTok{] }\CommentTok{\# Player range}
\NormalTok{    axes }\OperatorTok{=}\NormalTok{ []}
\NormalTok{    axes.append(plt.subplot(}\DecValTok{1}\NormalTok{,}\DecValTok{2}\NormalTok{,}\DecValTok{1}\NormalTok{))}
\NormalTok{    plt.title(}\StringTok{\textquotesingle{}No Usable Ace\textquotesingle{}}\NormalTok{)}
\NormalTok{    plt.imshow(v[prange[}\DecValTok{0}\NormalTok{]}\OperatorTok{{-}}\DecValTok{1}\NormalTok{:prange[}\DecValTok{1}\NormalTok{],:,}\DecValTok{0}\NormalTok{], }
\NormalTok{               vmin}\OperatorTok{={-}}\DecValTok{1}\NormalTok{,vmax}\OperatorTok{=}\DecValTok{1}\NormalTok{,}
\NormalTok{               origin}\OperatorTok{=}\StringTok{\textquotesingle{}lower\textquotesingle{}}\NormalTok{, }
\NormalTok{               extent}\OperatorTok{=}\NormalTok{(drange[}\DecValTok{0}\NormalTok{]}\OperatorTok{{-}}\FloatTok{0.5}\NormalTok{,}
\NormalTok{                       drange[}\DecValTok{1}\NormalTok{]}\OperatorTok{+}\FloatTok{0.5}\NormalTok{,}
\NormalTok{                       prange[}\DecValTok{0}\NormalTok{]}\OperatorTok{{-}}\FloatTok{0.5}\NormalTok{,}
\NormalTok{                       prange[}\DecValTok{1}\NormalTok{]}\OperatorTok{+}\FloatTok{0.5}\NormalTok{),}
\NormalTok{               cmap }\OperatorTok{=} \StringTok{\textquotesingle{}RdBu\textquotesingle{}}\NormalTok{)}
\NormalTok{    plt.xticks(np.arange(drange[}\DecValTok{0}\NormalTok{],drange[}\DecValTok{1}\NormalTok{]}\OperatorTok{+}\DecValTok{1}\NormalTok{,}\DecValTok{1}\NormalTok{))}
\NormalTok{    plt.yticks(np.arange(prange[}\DecValTok{0}\NormalTok{],prange[}\DecValTok{1}\NormalTok{]}\OperatorTok{+}\DecValTok{1}\NormalTok{,}\DecValTok{1}\NormalTok{))}
\NormalTok{    plt.ylabel(}\StringTok{\textquotesingle{}Player Total\textquotesingle{}}\NormalTok{)}
\NormalTok{    plt.xlabel(}\StringTok{\textquotesingle{}Dealer\textquotesingle{}}\NormalTok{)}
    
\NormalTok{    axes.append(plt.subplot(}\DecValTok{1}\NormalTok{,}\DecValTok{2}\NormalTok{,}\DecValTok{2}\NormalTok{))}
\NormalTok{    plt.title(}\StringTok{\textquotesingle{}Usable Ace\textquotesingle{}}\NormalTok{)}
\NormalTok{    im }\OperatorTok{=}\NormalTok{ plt.imshow(v[prange[}\DecValTok{0}\NormalTok{]}\OperatorTok{{-}}\DecValTok{1}\NormalTok{:prange[}\DecValTok{1}\NormalTok{],:,}\DecValTok{1}\NormalTok{], }
\NormalTok{               vmin}\OperatorTok{={-}}\DecValTok{1}\NormalTok{,vmax}\OperatorTok{=}\DecValTok{1}\NormalTok{,}
\NormalTok{               origin}\OperatorTok{=}\StringTok{\textquotesingle{}lower\textquotesingle{}}\NormalTok{, }
\NormalTok{               extent}\OperatorTok{=}\NormalTok{(drange[}\DecValTok{0}\NormalTok{]}\OperatorTok{{-}}\FloatTok{0.5}\NormalTok{,}
\NormalTok{                       drange[}\DecValTok{1}\NormalTok{]}\OperatorTok{+}\FloatTok{0.5}\NormalTok{,}
\NormalTok{                       prange[}\DecValTok{0}\NormalTok{]}\OperatorTok{{-}}\FloatTok{0.5}\NormalTok{,}
\NormalTok{                       prange[}\DecValTok{1}\NormalTok{]}\OperatorTok{+}\FloatTok{0.5}\NormalTok{),}
\NormalTok{               cmap }\OperatorTok{=} \StringTok{\textquotesingle{}RdBu\textquotesingle{}}\NormalTok{)}
\NormalTok{    plt.xticks(np.arange(drange[}\DecValTok{0}\NormalTok{],drange[}\DecValTok{1}\NormalTok{]}\OperatorTok{+}\DecValTok{1}\NormalTok{,}\DecValTok{1}\NormalTok{))}
\NormalTok{    plt.yticks(np.arange(prange[}\DecValTok{0}\NormalTok{],prange[}\DecValTok{1}\NormalTok{]}\OperatorTok{+}\DecValTok{1}\NormalTok{,}\DecValTok{1}\NormalTok{))}
\NormalTok{    plt.ylabel(}\StringTok{\textquotesingle{}Player Total\textquotesingle{}}\NormalTok{)}
\NormalTok{    plt.xlabel(}\StringTok{\textquotesingle{}Dealer\textquotesingle{}}\NormalTok{)}
\NormalTok{    cbar }\OperatorTok{=}\NormalTok{ plt.colorbar(im, ax}\OperatorTok{=}\NormalTok{axes)}
\NormalTok{    cbar.set\_label(}\StringTok{\textquotesingle{}Value (Expected Returns)\textquotesingle{}}\NormalTok{)}
\end{Highlighting}
\end{Shaded}

\subsection{Exercise 3 - Perform Monte Carlo
Control}\label{exercise-3---perform-monte-carlo-control}

\textbf{{[}40 points{]}}

Now it's time to actually implement a reinforcement learning strategy
that learns to play this version of Blackjack well, only through
trial-and-error learning. Here, you will develop your Monte Carlo
Control algorithm and evaluate its performance for our Blackjack-like
game.

\textbf{3.1.} Using Monte Carlo Control through policy iteration,
estimate the optimal policy for playing our modified blackjack game to
maximize rewards.

In doing this, use the following assumptions:

\begin{enumerate}
\def\labelenumi{\arabic{enumi}.}
\tightlist
\item
  Initialize the state value function and the action value function to
  all zeros
\item
  Keep a running tally of the number of times the agent visited each
  state and chose an action. \(N(s_t,a_t)\) is the number of times
  action \(a\) has been selected from state \(s\). You'll need this to
  compute the running average. You can implement an online average as:
  \(\bar{x}_{t} = \frac{1}{N}x_t + \frac{N-1}{N}\bar{x}_{t-1}\)
\item
  Use an \(\epsilon\)-greedy exploration strategy with
  \(\epsilon_t = \frac{N_0}{N_0 + N(s_t)}\), where we define
  \(N_0 = 100\). Vary \(N_0\) as needed. Varying \(0 \leq N_0 < \inf\)
  will determine the amount of exploration the algorithm performs where
  the lower \(N_0\) the less exploration and vice versa.
\end{enumerate}

Show your results by plotting the optimal state value function:
\(v^*(s) = \max_a q^*(s,a)\) and the optimal policy \(\pi^*(s)\). Create
plots for these similar to Sutton and Barto, Figure 5.2 in the 2018
edition (5.5 in the original edition) - sample code provided for the
plots. Your results SHOULD be very similar to the plots in that text
(although you will show your results with the player sum ranging from 4
to 21). For these plots include the following (note - code from the
previous section of this assignment for state value function plotting
and below for policy plotting are provided to help you to accomplish
these):

\begin{enumerate}
\def\labelenumi{\arabic{enumi}.}
\tightlist
\item
  When you have a useable ace, plot the value function with the dealer's
  card on the x-axis, and the player's sum on the y-axis, and use the
  colormap and \texttt{imshow} to plot the value function that
  corresponds with those states. Plot the state value corresponding to
  each state under the policy described above. The domain of your x and
  y axes should include all possible states (4 to 21 for the player sum,
  and 1 to 10 for the dealer's visible card).
\item
  Repeat (1) for the same states but without a usable ace.
\item
  Plot the optimal policy \(\pi^*(s)\) for the states with a usable ace
  (this plot can be an imshow plot with binary values - sample code
  provided).
\item
  Plot the optimal policy \(\pi^*(s)\) for the states without a usable
  ace (this plot can be an imshow plot with binary values - sample code
  provided).
\end{enumerate}

\textbf{3.2.} Plot the cumulative average return per episode vs the
number of episodes (your x-axis should be log-scaled to clearly see the
trend). What is the average return your control strategy was able to
achieve? You'll know your method is working if you see a steady rise in
your average returns over time.

\emph{Note on convergence: convergence of this algorithm is extremely
slow. You may need to let this run a few million episodes before the
policy starts to converge. You're not expected to get EXACTLY the
optimal policy, but it should be visibly close.}

\emph{Note on sample code: the code provided for these questions is
meant to be a helpful starting point - you are not required to fill it
out exactly or use all components, but it is meant to help you as you
begin thinking about this problem.}

\textbf{Sample Code}

\begin{Shaded}
\begin{Highlighting}[]
\KeywordTok{def}\NormalTok{ mc\_control(episodes):}
\NormalTok{    N\_player  }\OperatorTok{=} \DecValTok{32}
\NormalTok{    N\_dealer  }\OperatorTok{=} \DecValTok{10}
\NormalTok{    N\_ace     }\OperatorTok{=} \DecValTok{2}
\NormalTok{    N\_actions }\OperatorTok{=} \DecValTok{2}
    
\NormalTok{    avg\_return }\OperatorTok{=}\NormalTok{ []}
\NormalTok{    N          }\OperatorTok{=} \DecValTok{0}
\NormalTok{    N0         }\OperatorTok{=} \DecValTok{1} \CommentTok{\# Coefficient for controlling the level of exploration}
    
    \CommentTok{\# Initialize the state value function, v, and the action value function, q}
\NormalTok{    v }\OperatorTok{=}\NormalTok{ np.zeros((N\_player, N\_dealer, N\_ace))}
\NormalTok{    q }\OperatorTok{=}\NormalTok{ np.zeros((N\_player, N\_dealer, N\_ace, N\_actions))}
    
    \CommentTok{\# Initialize a variable counting the number of visits to each state}
    \CommentTok{\# For the state value function}
\NormalTok{    nv\_visits }\OperatorTok{=}\NormalTok{ np.zeros((N\_player, N\_dealer, N\_ace)) }
    \CommentTok{\# For the action value function}
\NormalTok{    nq\_visits }\OperatorTok{=}\NormalTok{ np.zeros((N\_player, N\_dealer, N\_ace, N\_actions)) }
    
    \CommentTok{\# Initialize the policy to all always "stay" (all zeros)}
\NormalTok{    pi }\OperatorTok{=}\NormalTok{ np.zeros((N\_player, N\_dealer, N\_ace))}
    
    \CommentTok{\# Initialize the game}
\NormalTok{    B  }\OperatorTok{=}\NormalTok{ Blackjack()}

    \CommentTok{\# HELPER FUNCTIONS}
    \CommentTok{\# Convert the current state into a set of indices for the }
    \CommentTok{\#  state value function}
    \KeywordTok{def}\NormalTok{ state\_to\_index\_v(s):}
\NormalTok{        ace }\OperatorTok{=} \DecValTok{0}
        \ControlFlowTok{if}\NormalTok{ s[}\DecValTok{2}\NormalTok{]:}
\NormalTok{            ace }\OperatorTok{=} \DecValTok{1}
\NormalTok{        index }\OperatorTok{=}\NormalTok{ [s[}\DecValTok{0}\NormalTok{]}\OperatorTok{{-}}\DecValTok{1}\NormalTok{, s[}\DecValTok{1}\NormalTok{]}\OperatorTok{{-}}\DecValTok{1}\NormalTok{, ace]}
        \ControlFlowTok{return}\NormalTok{ index}
    
    \CommentTok{\# Convert the current state into a set of indices for the }
    \CommentTok{\#  action value function}
    \KeywordTok{def}\NormalTok{ state\_to\_index\_av(s,a):}
\NormalTok{        ace }\OperatorTok{=} \DecValTok{0}
        \ControlFlowTok{if}\NormalTok{ s[}\DecValTok{2}\NormalTok{]:}
\NormalTok{            ace }\OperatorTok{=} \DecValTok{1}
\NormalTok{        index }\OperatorTok{=}\NormalTok{ [s[}\DecValTok{0}\NormalTok{]}\OperatorTok{{-}}\DecValTok{1}\NormalTok{, s[}\DecValTok{1}\NormalTok{]}\OperatorTok{{-}}\DecValTok{1}\NormalTok{, ace, }\BuiltInTok{int}\NormalTok{(a)]}
        \ControlFlowTok{return}\NormalTok{ index}
    
    \CommentTok{\# Choose an action based on the policy, pi, and the current state, s, with }
    \CommentTok{\#  epsilon{-}greedy exploration}
    \KeywordTok{def}\NormalTok{ choose\_action(pi,s):}
        \CommentTok{\#\#\#\#\#\#\#\#\#\#\#\#\#\#\#\#\#\#\#\#\#\#\#}
        \CommentTok{\# FILL IN THIS FUNCTION}
        \CommentTok{\#\#\#\#\#\#\#\#\#\#\#\#\#\#\#\#\#\#\#\#\#\#\#}
        \CommentTok{\# Outputs an action}
    
    \CommentTok{\# {-}{-}{-}{-}{-}{-}{-}{-}{-}{-}{-}{-}{-}{-}{-}{-}{-}{-}{-}{-}{-}{-}{-}{-}{-}{-}{-}{-}{-}{-}{-}{-}{-}{-}{-}{-}{-}{-}{-}{-}{-}{-}{-}{-}{-}{-}{-}{-}{-}{-}{-}{-}{-}{-}{-}{-}{-}{-}{-}{-}{-}{-}{-}{-}{-}{-}{-}{-}{-}{-}}
    \CommentTok{\# Run episodes of MC Control    }
    \CommentTok{\# {-}{-}{-}{-}{-}{-}{-}{-}{-}{-}{-}{-}{-}{-}{-}{-}{-}{-}{-}{-}{-}{-}{-}{-}{-}{-}{-}{-}{-}{-}{-}{-}{-}{-}{-}{-}{-}{-}{-}{-}{-}{-}{-}{-}{-}{-}{-}{-}{-}{-}{-}{-}{-}{-}{-}{-}{-}{-}{-}{-}{-}{-}{-}{-}{-}{-}{-}{-}{-}{-}}
    \ControlFlowTok{for}\NormalTok{ episode }\KeywordTok{in} \BuiltInTok{range}\NormalTok{(episodes):}
        \CommentTok{\# Play a hand a blackjack       }
        
        \CommentTok{\# Update average return count}
        
        \CommentTok{\# Update the state value function}
        
        \CommentTok{\# Update the action value function }
        
        \CommentTok{\# Update the policy}

    \ControlFlowTok{return}\NormalTok{ (v, pi, avg\_return)}
\end{Highlighting}
\end{Shaded}

\begin{Shaded}
\begin{Highlighting}[]
\CommentTok{\#{-}{-}{-}{-}{-}{-}{-}{-}{-}{-}{-}{-}{-}{-}{-}{-}{-}{-}{-}{-}{-}{-}{-}{-}{-}{-}{-}{-}{-}{-}{-}{-}{-}{-}{-}{-}{-}{-}{-}{-}{-}{-}{-}{-}{-}{-}{-}{-}{-}}
\CommentTok{\# Sample code for plotting your policy}
\CommentTok{\#{-}{-}{-}{-}{-}{-}{-}{-}{-}{-}{-}{-}{-}{-}{-}{-}{-}{-}{-}{-}{-}{-}{-}{-}{-}{-}{-}{-}{-}{-}{-}{-}{-}{-}{-}{-}{-}{-}{-}{-}{-}{-}{-}{-}{-}{-}{-}{-}{-}}
\KeywordTok{def}\NormalTok{ plot\_policy(pi):}
    \CommentTok{\# Plot the policy}
\NormalTok{    drange }\OperatorTok{=}\NormalTok{ [}\DecValTok{1}\NormalTok{,}\DecValTok{10}\NormalTok{]}
\NormalTok{    prange }\OperatorTok{=}\NormalTok{ [}\DecValTok{4}\NormalTok{,}\DecValTok{21}\NormalTok{]}
\NormalTok{    plt.figure(figsize}\OperatorTok{=}\NormalTok{(}\DecValTok{10}\NormalTok{,}\DecValTok{7}\NormalTok{))}
\NormalTok{    axes }\OperatorTok{=}\NormalTok{ []}
\NormalTok{    axes.append(plt.subplot(}\DecValTok{1}\NormalTok{,}\DecValTok{2}\NormalTok{,}\DecValTok{1}\NormalTok{))}
\NormalTok{    plt.title(}\StringTok{\textquotesingle{}No Usable Ace\textquotesingle{}}\NormalTok{)}
\NormalTok{    plt.imshow(pi[prange[}\DecValTok{0}\NormalTok{]}\OperatorTok{{-}}\DecValTok{1}\NormalTok{:prange[}\DecValTok{1}\NormalTok{],:,}\DecValTok{0}\NormalTok{], }
\NormalTok{               vmin}\OperatorTok{=}\DecValTok{0}\NormalTok{,vmax}\OperatorTok{=}\DecValTok{1}\NormalTok{,}
\NormalTok{               origin}\OperatorTok{=}\StringTok{\textquotesingle{}lower\textquotesingle{}}\NormalTok{, }
\NormalTok{               extent}\OperatorTok{=}\NormalTok{(drange[}\DecValTok{0}\NormalTok{]}\OperatorTok{{-}}\FloatTok{0.5}\NormalTok{,}
\NormalTok{                       drange[}\DecValTok{1}\NormalTok{]}\OperatorTok{+}\FloatTok{0.5}\NormalTok{,}
\NormalTok{                       prange[}\DecValTok{0}\NormalTok{]}\OperatorTok{{-}}\FloatTok{0.5}\NormalTok{,}
\NormalTok{                       prange[}\DecValTok{1}\NormalTok{]}\OperatorTok{+}\FloatTok{0.5}\NormalTok{),}
\NormalTok{               cmap }\OperatorTok{=} \StringTok{\textquotesingle{}binary\_r\textquotesingle{}}\NormalTok{)}
\NormalTok{    plt.xticks(np.arange(drange[}\DecValTok{0}\NormalTok{],drange[}\DecValTok{1}\NormalTok{]}\OperatorTok{+}\DecValTok{1}\NormalTok{,}\DecValTok{1}\NormalTok{))}
\NormalTok{    plt.yticks(np.arange(prange[}\DecValTok{0}\NormalTok{],prange[}\DecValTok{1}\NormalTok{]}\OperatorTok{+}\DecValTok{1}\NormalTok{,}\DecValTok{1}\NormalTok{))}
\NormalTok{    plt.ylabel(}\StringTok{\textquotesingle{}Player Total\textquotesingle{}}\NormalTok{)}
\NormalTok{    plt.xlabel(}\StringTok{\textquotesingle{}Dealer\textquotesingle{}}\NormalTok{)}
    
\NormalTok{    axes.append(plt.subplot(}\DecValTok{1}\NormalTok{,}\DecValTok{2}\NormalTok{,}\DecValTok{2}\NormalTok{))}
\NormalTok{    plt.title(}\StringTok{\textquotesingle{}Usable Ace\textquotesingle{}}\NormalTok{)}
\NormalTok{    im }\OperatorTok{=}\NormalTok{ plt.imshow(pi[prange[}\DecValTok{0}\NormalTok{]}\OperatorTok{{-}}\DecValTok{1}\NormalTok{:prange[}\DecValTok{1}\NormalTok{],:,}\DecValTok{1}\NormalTok{], }
\NormalTok{               vmin}\OperatorTok{=}\DecValTok{0}\NormalTok{,vmax}\OperatorTok{=}\DecValTok{1}\NormalTok{,}
\NormalTok{               origin}\OperatorTok{=}\StringTok{\textquotesingle{}lower\textquotesingle{}}\NormalTok{, }
\NormalTok{               extent}\OperatorTok{=}\NormalTok{(drange[}\DecValTok{0}\NormalTok{]}\OperatorTok{{-}}\FloatTok{0.5}\NormalTok{,}
\NormalTok{                       drange[}\DecValTok{1}\NormalTok{]}\OperatorTok{+}\FloatTok{0.5}\NormalTok{,}
\NormalTok{                       prange[}\DecValTok{0}\NormalTok{]}\OperatorTok{{-}}\FloatTok{0.5}\NormalTok{,}
\NormalTok{                       prange[}\DecValTok{1}\NormalTok{]}\OperatorTok{+}\FloatTok{0.5}\NormalTok{),}
\NormalTok{               cmap }\OperatorTok{=} \StringTok{\textquotesingle{}binary\_r\textquotesingle{}}\NormalTok{)}
\NormalTok{    plt.xticks(np.arange(drange[}\DecValTok{0}\NormalTok{],drange[}\DecValTok{1}\NormalTok{]}\OperatorTok{+}\DecValTok{1}\NormalTok{,}\DecValTok{1}\NormalTok{))}
\NormalTok{    plt.yticks(np.arange(prange[}\DecValTok{0}\NormalTok{],prange[}\DecValTok{1}\NormalTok{]}\OperatorTok{+}\DecValTok{1}\NormalTok{,}\DecValTok{1}\NormalTok{))}
\NormalTok{    plt.ylabel(}\StringTok{\textquotesingle{}Player Total\textquotesingle{}}\NormalTok{)}
\NormalTok{    plt.xlabel(}\StringTok{\textquotesingle{}Dealer\textquotesingle{}}\NormalTok{)}
\NormalTok{    cbar }\OperatorTok{=}\NormalTok{ plt.colorbar(im, ax}\OperatorTok{=}\NormalTok{axes)}
\NormalTok{    cbar.set\_label(}\StringTok{\textquotesingle{}Action (Stay = 0, Hit = 1)\textquotesingle{}}\NormalTok{)}
\end{Highlighting}
\end{Shaded}

\subsection{Exercise 4 - Discuss your
findings}\label{exercise-4---discuss-your-findings}

\textbf{{[}5 points{]}}

Compare the performance of your human control policy, in question 1, the
naive policy from question 2, and the optimal control policy in question
3.

\textbf{4.1.} Which performs best? What was different about the policies
developed for each and how may that have contributed to their
comparative advantages?

\textbf{4.2.} Could you have created a better policy if you knew the
full Markov Decision Process for this environment? Why or why not?
\emph{(assume the policy estimated from your MC control algorithm had
\textbf{fully converged})}




\end{document}
