% Options for packages loaded elsewhere
\PassOptionsToPackage{unicode}{hyperref}
\PassOptionsToPackage{hyphens}{url}
\PassOptionsToPackage{dvipsnames,svgnames,x11names}{xcolor}
%
\documentclass[
  letterpaper,
  DIV=11,
  numbers=noendperiod]{scrartcl}

\usepackage{amsmath,amssymb}
\usepackage{iftex}
\ifPDFTeX
  \usepackage[T1]{fontenc}
  \usepackage[utf8]{inputenc}
  \usepackage{textcomp} % provide euro and other symbols
\else % if luatex or xetex
  \usepackage{unicode-math}
  \defaultfontfeatures{Scale=MatchLowercase}
  \defaultfontfeatures[\rmfamily]{Ligatures=TeX,Scale=1}
\fi
\usepackage{lmodern}
\ifPDFTeX\else  
    % xetex/luatex font selection
\fi
% Use upquote if available, for straight quotes in verbatim environments
\IfFileExists{upquote.sty}{\usepackage{upquote}}{}
\IfFileExists{microtype.sty}{% use microtype if available
  \usepackage[]{microtype}
  \UseMicrotypeSet[protrusion]{basicmath} % disable protrusion for tt fonts
}{}
\makeatletter
\@ifundefined{KOMAClassName}{% if non-KOMA class
  \IfFileExists{parskip.sty}{%
    \usepackage{parskip}
  }{% else
    \setlength{\parindent}{0pt}
    \setlength{\parskip}{6pt plus 2pt minus 1pt}}
}{% if KOMA class
  \KOMAoptions{parskip=half}}
\makeatother
\usepackage{xcolor}
\setlength{\emergencystretch}{3em} % prevent overfull lines
\setcounter{secnumdepth}{-\maxdimen} % remove section numbering
% Make \paragraph and \subparagraph free-standing
\makeatletter
\ifx\paragraph\undefined\else
  \let\oldparagraph\paragraph
  \renewcommand{\paragraph}{
    \@ifstar
      \xxxParagraphStar
      \xxxParagraphNoStar
  }
  \newcommand{\xxxParagraphStar}[1]{\oldparagraph*{#1}\mbox{}}
  \newcommand{\xxxParagraphNoStar}[1]{\oldparagraph{#1}\mbox{}}
\fi
\ifx\subparagraph\undefined\else
  \let\oldsubparagraph\subparagraph
  \renewcommand{\subparagraph}{
    \@ifstar
      \xxxSubParagraphStar
      \xxxSubParagraphNoStar
  }
  \newcommand{\xxxSubParagraphStar}[1]{\oldsubparagraph*{#1}\mbox{}}
  \newcommand{\xxxSubParagraphNoStar}[1]{\oldsubparagraph{#1}\mbox{}}
\fi
\makeatother

\usepackage{color}
\usepackage{fancyvrb}
\newcommand{\VerbBar}{|}
\newcommand{\VERB}{\Verb[commandchars=\\\{\}]}
\DefineVerbatimEnvironment{Highlighting}{Verbatim}{commandchars=\\\{\}}
% Add ',fontsize=\small' for more characters per line
\usepackage{framed}
\definecolor{shadecolor}{RGB}{241,243,245}
\newenvironment{Shaded}{\begin{snugshade}}{\end{snugshade}}
\newcommand{\AlertTok}[1]{\textcolor[rgb]{0.68,0.00,0.00}{#1}}
\newcommand{\AnnotationTok}[1]{\textcolor[rgb]{0.37,0.37,0.37}{#1}}
\newcommand{\AttributeTok}[1]{\textcolor[rgb]{0.40,0.45,0.13}{#1}}
\newcommand{\BaseNTok}[1]{\textcolor[rgb]{0.68,0.00,0.00}{#1}}
\newcommand{\BuiltInTok}[1]{\textcolor[rgb]{0.00,0.23,0.31}{#1}}
\newcommand{\CharTok}[1]{\textcolor[rgb]{0.13,0.47,0.30}{#1}}
\newcommand{\CommentTok}[1]{\textcolor[rgb]{0.37,0.37,0.37}{#1}}
\newcommand{\CommentVarTok}[1]{\textcolor[rgb]{0.37,0.37,0.37}{\textit{#1}}}
\newcommand{\ConstantTok}[1]{\textcolor[rgb]{0.56,0.35,0.01}{#1}}
\newcommand{\ControlFlowTok}[1]{\textcolor[rgb]{0.00,0.23,0.31}{\textbf{#1}}}
\newcommand{\DataTypeTok}[1]{\textcolor[rgb]{0.68,0.00,0.00}{#1}}
\newcommand{\DecValTok}[1]{\textcolor[rgb]{0.68,0.00,0.00}{#1}}
\newcommand{\DocumentationTok}[1]{\textcolor[rgb]{0.37,0.37,0.37}{\textit{#1}}}
\newcommand{\ErrorTok}[1]{\textcolor[rgb]{0.68,0.00,0.00}{#1}}
\newcommand{\ExtensionTok}[1]{\textcolor[rgb]{0.00,0.23,0.31}{#1}}
\newcommand{\FloatTok}[1]{\textcolor[rgb]{0.68,0.00,0.00}{#1}}
\newcommand{\FunctionTok}[1]{\textcolor[rgb]{0.28,0.35,0.67}{#1}}
\newcommand{\ImportTok}[1]{\textcolor[rgb]{0.00,0.46,0.62}{#1}}
\newcommand{\InformationTok}[1]{\textcolor[rgb]{0.37,0.37,0.37}{#1}}
\newcommand{\KeywordTok}[1]{\textcolor[rgb]{0.00,0.23,0.31}{\textbf{#1}}}
\newcommand{\NormalTok}[1]{\textcolor[rgb]{0.00,0.23,0.31}{#1}}
\newcommand{\OperatorTok}[1]{\textcolor[rgb]{0.37,0.37,0.37}{#1}}
\newcommand{\OtherTok}[1]{\textcolor[rgb]{0.00,0.23,0.31}{#1}}
\newcommand{\PreprocessorTok}[1]{\textcolor[rgb]{0.68,0.00,0.00}{#1}}
\newcommand{\RegionMarkerTok}[1]{\textcolor[rgb]{0.00,0.23,0.31}{#1}}
\newcommand{\SpecialCharTok}[1]{\textcolor[rgb]{0.37,0.37,0.37}{#1}}
\newcommand{\SpecialStringTok}[1]{\textcolor[rgb]{0.13,0.47,0.30}{#1}}
\newcommand{\StringTok}[1]{\textcolor[rgb]{0.13,0.47,0.30}{#1}}
\newcommand{\VariableTok}[1]{\textcolor[rgb]{0.07,0.07,0.07}{#1}}
\newcommand{\VerbatimStringTok}[1]{\textcolor[rgb]{0.13,0.47,0.30}{#1}}
\newcommand{\WarningTok}[1]{\textcolor[rgb]{0.37,0.37,0.37}{\textit{#1}}}

\providecommand{\tightlist}{%
  \setlength{\itemsep}{0pt}\setlength{\parskip}{0pt}}\usepackage{longtable,booktabs,array}
\usepackage{calc} % for calculating minipage widths
% Correct order of tables after \paragraph or \subparagraph
\usepackage{etoolbox}
\makeatletter
\patchcmd\longtable{\par}{\if@noskipsec\mbox{}\fi\par}{}{}
\makeatother
% Allow footnotes in longtable head/foot
\IfFileExists{footnotehyper.sty}{\usepackage{footnotehyper}}{\usepackage{footnote}}
\makesavenoteenv{longtable}
\usepackage{graphicx}
\makeatletter
\newsavebox\pandoc@box
\newcommand*\pandocbounded[1]{% scales image to fit in text height/width
  \sbox\pandoc@box{#1}%
  \Gscale@div\@tempa{\textheight}{\dimexpr\ht\pandoc@box+\dp\pandoc@box\relax}%
  \Gscale@div\@tempb{\linewidth}{\wd\pandoc@box}%
  \ifdim\@tempb\p@<\@tempa\p@\let\@tempa\@tempb\fi% select the smaller of both
  \ifdim\@tempa\p@<\p@\scalebox{\@tempa}{\usebox\pandoc@box}%
  \else\usebox{\pandoc@box}%
  \fi%
}
% Set default figure placement to htbp
\def\fps@figure{htbp}
\makeatother

\KOMAoption{captions}{tableheading}
\makeatletter
\@ifpackageloaded{tcolorbox}{}{\usepackage[skins,breakable]{tcolorbox}}
\@ifpackageloaded{fontawesome5}{}{\usepackage{fontawesome5}}
\definecolor{quarto-callout-color}{HTML}{909090}
\definecolor{quarto-callout-note-color}{HTML}{0758E5}
\definecolor{quarto-callout-important-color}{HTML}{CC1914}
\definecolor{quarto-callout-warning-color}{HTML}{EB9113}
\definecolor{quarto-callout-tip-color}{HTML}{00A047}
\definecolor{quarto-callout-caution-color}{HTML}{FC5300}
\definecolor{quarto-callout-color-frame}{HTML}{acacac}
\definecolor{quarto-callout-note-color-frame}{HTML}{4582ec}
\definecolor{quarto-callout-important-color-frame}{HTML}{d9534f}
\definecolor{quarto-callout-warning-color-frame}{HTML}{f0ad4e}
\definecolor{quarto-callout-tip-color-frame}{HTML}{02b875}
\definecolor{quarto-callout-caution-color-frame}{HTML}{fd7e14}
\makeatother
\makeatletter
\@ifpackageloaded{caption}{}{\usepackage{caption}}
\AtBeginDocument{%
\ifdefined\contentsname
  \renewcommand*\contentsname{Table of contents}
\else
  \newcommand\contentsname{Table of contents}
\fi
\ifdefined\listfigurename
  \renewcommand*\listfigurename{List of Figures}
\else
  \newcommand\listfigurename{List of Figures}
\fi
\ifdefined\listtablename
  \renewcommand*\listtablename{List of Tables}
\else
  \newcommand\listtablename{List of Tables}
\fi
\ifdefined\figurename
  \renewcommand*\figurename{Figure}
\else
  \newcommand\figurename{Figure}
\fi
\ifdefined\tablename
  \renewcommand*\tablename{Table}
\else
  \newcommand\tablename{Table}
\fi
}
\@ifpackageloaded{float}{}{\usepackage{float}}
\floatstyle{ruled}
\@ifundefined{c@chapter}{\newfloat{codelisting}{h}{lop}}{\newfloat{codelisting}{h}{lop}[chapter]}
\floatname{codelisting}{Listing}
\newcommand*\listoflistings{\listof{codelisting}{List of Listings}}
\makeatother
\makeatletter
\makeatother
\makeatletter
\@ifpackageloaded{caption}{}{\usepackage{caption}}
\@ifpackageloaded{subcaption}{}{\usepackage{subcaption}}
\makeatother

\usepackage{bookmark}

\IfFileExists{xurl.sty}{\usepackage{xurl}}{} % add URL line breaks if available
\urlstyle{same} % disable monospaced font for URLs
\hypersetup{
  pdftitle={Assignment 5},
  pdfauthor={Kyle Bradbury},
  colorlinks=true,
  linkcolor={blue},
  filecolor={Maroon},
  citecolor={Blue},
  urlcolor={Blue},
  pdfcreator={LaTeX via pandoc}}


\title{Assignment 5}
\usepackage{etoolbox}
\makeatletter
\providecommand{\subtitle}[1]{% add subtitle to \maketitle
  \apptocmd{\@title}{\par {\large #1 \par}}{}{}
}
\makeatother
\subtitle{Kaggle Competition and Unsupervised Learning}
\author{Kyle Bradbury}
\date{2025-03-05}

\begin{document}
\maketitle

\renewcommand*\contentsname{Table of contents}
{
\hypersetup{linkcolor=}
\setcounter{tocdepth}{1}
\tableofcontents
}

\subsection{Instructions}\label{instructions}

\emph{Instructions for all assignments can be found
\href{https://kylebradbury.github.io/ids705/notebooks/assignment_instructions.html}{here}.
Note: this assignment falls under collaboration Mode 2: Individual
Assignment -- Collaboration Permitted. Please refer to the syllabus for
additional information. Please be sure to list the names of any students
that you worked with on this assignment. Total points in the assignment
add up to 90; an additional 10 points are allocated to professionalism
and presentation quality.}

\subsection{Learning objectives}\label{learning-objectives}

Through completing this assignment you will be able to\ldots{}

\begin{enumerate}
\def\labelenumi{\arabic{enumi}.}
\tightlist
\item
  Apply the full supervised machine learning pipeline of preprocessing,
  model selection, model performance evaluation and comparison, and
  model application to a real-world scale dataset
\item
  Apply clustering techniques to a variety of datasets with diverse
  distributional properties, gaining an understanding of their strengths
  and weaknesses and how to tune model parameters
\item
  Apply PCA and t-SNE for performing dimensionality reduction and data
  visualization
\end{enumerate}

\subsection{Exercise 1 - Kaggle Classification
Competition}\label{exercise-1---kaggle-classification-competition}

\textbf{{[}40 points{]}}

You've learned a great deal about supervised learning and now it's time
to bring together all that you've learned. You will be competing in a
Kaggle Competition along with the rest of the class! Your goal is to
predict hotel reservation cancellations based on a number of potentially
related factors such as lead time on the booking, time of year, type of
room, special requests made, number of children, etc. While you will be
asked to take certain steps along the way to your submission, you're
encouraged to try creative solutions to this problem and your choices
are wide open for you to make your decisions on how to best make the
predictions.

\begin{tcolorbox}[enhanced jigsaw, colframe=quarto-callout-important-color-frame, rightrule=.15mm, opacityback=0, breakable, bottomtitle=1mm, opacitybacktitle=0.6, left=2mm, colbacktitle=quarto-callout-important-color!10!white, arc=.35mm, title=\textcolor{quarto-callout-important-color}{\faExclamation}\hspace{0.5em}{Important note}, toprule=.15mm, toptitle=1mm, titlerule=0mm, bottomrule=.15mm, coltitle=black, leftrule=.75mm, colback=white]

Follow the link posted on Ed to register for the competition. You can
view the public leaderboard anytime at the Kaggle website (see the Ed
post).

\end{tcolorbox}

\textbf{The Data}. The dataset is provided as \texttt{a5\_q1.pkl} which
is a pickle file format, which allows you to load the data directly
using the code below; the data can be downloaded from the Kaggle
competition website (see Ed Discussions for the link). A data dictionary
for the project can be found
\href{https://github.com/rfordatascience/tidytuesday/blob/master/data/2020/2020-02-11/readme.md}{here}
and the original paper that describes the dataset can be found
\href{https://www.sciencedirect.com/science/article/pii/S2352340918315191}{here}.
When you load the data, 5 matrices are provided
\texttt{X\_train\_original}, \texttt{y\_train}, and
\texttt{X\_test\_original}, which are the original, unprocessed features
and labels for the training set and the test features (the test labels
are not provided - that's what you're predicting). Additionally,
\texttt{X\_train\_ohe} and \texttt{X\_test\_ohe} are provided which are
one-hot-encoded (OHE) versions of the data. The OHE versions OHE
processed every categorical variable. This is provided for convenience
if you find it helpful, but you're welcome to reprocess the original
data other ways if your prefer.

\textbf{Scoring}. You will need to achieve a minimum acceptable level of
performance to demonstrate proficiency with using these supervised
learning techniques. Beyond that, it's an open competition and scoring
in the top three places of the \emph{private leaderboard} will result in
\textbf{3, 2, and 1 bonus points in this assignment, respectively} (and
the pride of the class!). Note: the Kaggle leaderboard has a public and
private component. The public component is viewable throughout the
competition, but the private leaderboard is revealed at the end. When
you make a submission, you immediately see your submission on the public
leaderboard, but that only represents scoring on a fraction of the total
collection of test data, the rest remains hidden until the end of the
competition to prevent overfitting to the test data through repeated
submissions. You will be be allowed to hand-select two eligible
submissions for private score, or by default your best two public
scoring submissions will be selected for private scoring.

\subsubsection{Requirements:}\label{requirements}

\textbf{1.1. Explore your data.} Review and understand your data. Look
at it; read up on what the features represent; think through the
application domain; visualize statistics from the paper data to
understand any key relationships. \textbf{There is no output required
for this question}, but you are encouraged to explore the data
personally before going further.

\textbf{1.2. Preprocess your data.} Preprocess your data so it's ready
for use for classification and describe what you did and why you did it.
Preprocessing may include: normalizing data, handling missing or
erroneous values, separating out a validation dataset, preparing
categorical variables through one-hot-encoding, etc. To make one step in
this process easier, you're provided with a one-hot-encoded version of
the data already.

\begin{itemize}
\tightlist
\item
  Comment on each type of preprocessing that you apply and both how and
  why you apply it.
\end{itemize}

\textbf{1.3. Select, train, and compare models.} Fit at least 5 models
to the data. Some of these can be experiments with different
hyperparameter-tuned versions of the same model, although all 5 should
not be the same type of model. There are no constraints on the types of
models, but you're encouraged to explore examples we've discussed in
class including:

\begin{enumerate}
\def\labelenumi{\arabic{enumi}.}
\tightlist
\item
  Logistic regression
\item
  K-nearest neighbors
\item
  Random Forests
\item
  Neural networks
\item
  Support Vector Machines
\item
  Ensembles of models (e.g.~model bagging, boosting, or stacking).
  \texttt{Scikit-learn} offers a number of tools for assisting with this
  including those for
  \href{https://scikit-learn.org/stable/modules/generated/sklearn.ensemble.BaggingClassifier.html\#sklearn.ensemble.BaggingClassifier}{bagging},
  \href{https://scikit-learn.org/stable/modules/generated/sklearn.ensemble.GradientBoostingClassifier.html}{boosting},
  and
  \href{https://scikit-learn.org/stable/modules/generated/sklearn.ensemble.StackingClassifier.html}{stacking}.
  You're also welcome to explore options beyond the \texttt{sklean}
  universe; for example, some of you may have heard of
  \href{https://github.com/dmlc/xgboost}{XGBoost} which is a very fast
  implementation of gradient boosted decision trees that also allows for
  parallelization.
\end{enumerate}

When selecting models, be aware that some models may take far longer
than others to train. Monitor your output and plan your time
accordingly.

Assess the classification performance AND computational efficiency of
the models you selected:

\begin{itemize}
\item
  Plot the ROC curves and PR curves for your models in two plots: one of
  ROC curves and one of PR curves. For each of these two plots, compare
  the performance of the models you selected above and trained on the
  training data, evaluating them on the validation data. Be sure to plot
  the line representing random guessing on each plot. You should plot
  all of the model's ROC curves on a single plot and the PR curves on a
  single plot. One of the models should also be your BEST performing
  submission on the Kaggle public leaderboard (see below). In the
  legends of each, include the area under the curve for each model
  (limit to 3 significant figures). For the ROC curve, this is the AUC;
  for the PR curve, this is the average precision (AP).
\item
  As you train and validate each model time how long it takes to train
  and validate in each case and create a plot that shows both the
  training and prediction time for each model included in the ROC and PR
  curves.
\item
  Describe:

  \begin{itemize}
  \tightlist
  \item
    Your process of model selection and hyperparameter tuning
  \item
    Which model performed best and your process for
    identifying/selecting it
  \end{itemize}
\end{itemize}

\textbf{1.4. Apply your model ``in practice''.} Make \emph{at least} 5
submissions of different model results to the competition (more
submissions are encouraged and you can submit up to 5 per day!). These
do not need to be the same that you report on above, but you should
select your \emph{most competitive} models.

\begin{itemize}
\tightlist
\item
  Produce submissions by applying your model on the test data.
\item
  Be sure to RETRAIN YOUR MODEL ON ALL LABELED TRAINING AND VALIDATION
  DATA before making your predictions on the test data for submission.
  This will help to maximize your performance on the test data.
\item
  In order to get full credit on this problem you must achieve an AUC on
  the Kaggle public leaderboard above the ``Benchmark'' score on the
  public leaderboard.
\end{itemize}

\subsubsection{Guidance}\label{guidance}

\begin{enumerate}
\def\labelenumi{\arabic{enumi}.}
\tightlist
\item
  \textbf{Preprocessing}. You may need to preprocess the data for some
  of these models to perform well (scaling inputs or reducing
  dimensionality). Some of this preprocessing may differ from model to
  model to achieve the best performance. A helpful tool for creating
  such preprocessing and model fitting pipelines is the sklearn
  \texttt{pipeline} module which lets you group a series of processing
  steps together.
\item
  \textbf{Hyperparameters}. Hyperparameters may need to be tuned for
  some of the model you use. You may want to perform hyperparameter
  tuning for some of the models. If you experiment with different
  hyperparameters that include many model runs, you may want to apply
  them to a small subsample of your overall data before running it on
  the larger training set to be time efficient (if you do, just make
  sure to ensure your selected subset is representative of the rest of
  your data).
\item
  \textbf{Validation data}. You're encouraged to create your own
  validation dataset for comparing model performance; without this,
  there's a significant likelihood of overfitting to the data. A common
  choice of the split is 80\% training, 20\% validation. Before you make
  your final predictions on the test data, be sure to retrain your model
  on the entire dataset.
\item
  \textbf{Training time}. This is a larger dataset than you've worked
  with previously in this class, so training times may be higher that
  what you've experienced in the past. Plan ahead and get your model
  pipeline working early so you can experiment with the models you use
  for this problem and have time to let them run.
\end{enumerate}

\subsubsection{Starter code}\label{starter-code}

Below is some code for (1) loading the data and (2) once you have
predictions in the form of confidence scores for those classifiers, to
produce submission files for Kaggle.

\begin{Shaded}
\begin{Highlighting}[]
\ImportTok{import}\NormalTok{ pandas }\ImportTok{as}\NormalTok{ pd}
\ImportTok{import}\NormalTok{ numpy }\ImportTok{as}\NormalTok{ np}
\ImportTok{import}\NormalTok{ pickle}

\CommentTok{\#\#\#\#\#\#\#\#\#\#\#\#\#\#\#\#\#\#\#\#\#\#\#\#\#\#\#\#\#\#\#\#}
\CommentTok{\# Load the data}
\CommentTok{\#\#\#\#\#\#\#\#\#\#\#\#\#\#\#\#\#\#\#\#\#\#\#\#\#\#\#\#\#\#\#\#}
\NormalTok{data }\OperatorTok{=}\NormalTok{ pd.read\_pickle(}\StringTok{"./data/a5\_q1.pkl"}\NormalTok{)}

\NormalTok{y\_train }\OperatorTok{=}\NormalTok{ data[}\StringTok{\textquotesingle{}y\_train\textquotesingle{}}\NormalTok{]}
\NormalTok{X\_train\_original }\OperatorTok{=}\NormalTok{ data[}\StringTok{\textquotesingle{}X\_train\textquotesingle{}}\NormalTok{] }\CommentTok{\# Original dataset}
\NormalTok{X\_train\_ohe }\OperatorTok{=}\NormalTok{ data[}\StringTok{\textquotesingle{}X\_train\_ohe\textquotesingle{}}\NormalTok{]  }\CommentTok{\# One{-}hot{-}encoded dataset}

\NormalTok{X\_test\_original }\OperatorTok{=}\NormalTok{ data[}\StringTok{\textquotesingle{}X\_test\textquotesingle{}}\NormalTok{]}
\NormalTok{X\_test\_ohe }\OperatorTok{=}\NormalTok{ data[}\StringTok{\textquotesingle{}X\_test\_ohe\textquotesingle{}}\NormalTok{]}

\CommentTok{\#\#\#\#\#\#\#\#\#\#\#\#\#\#\#\#\#\#\#\#\#\#\#\#\#\#\#\#\#\#\#\#}
\CommentTok{\# Produce submission}
\CommentTok{\#\#\#\#\#\#\#\#\#\#\#\#\#\#\#\#\#\#\#\#\#\#\#\#\#\#\#\#\#\#\#\#}

\KeywordTok{def}\NormalTok{ create\_submission(confidence\_scores, save\_path):}
    \CommentTok{\textquotesingle{}\textquotesingle{}\textquotesingle{}Creates an output file of submissions for Kaggle}
\CommentTok{    }
\CommentTok{    Parameters}
\CommentTok{    {-}{-}{-}{-}{-}{-}{-}{-}{-}{-}}
\CommentTok{    confidence\_scores : list or numpy array}
\CommentTok{        Confidence scores (from predict\_proba methods from classifiers) or}
\CommentTok{        binary predictions (only recommended in cases when predict\_proba is }
\CommentTok{        not available)}
\CommentTok{    save\_path : string}
\CommentTok{        File path for where to save the submission file.}
\CommentTok{    }
\CommentTok{    Example:}
\CommentTok{    create\_submission(my\_confidence\_scores, \textquotesingle{}./data/submission.csv\textquotesingle{})}

\CommentTok{    \textquotesingle{}\textquotesingle{}\textquotesingle{}}
    \ImportTok{import}\NormalTok{ pandas }\ImportTok{as}\NormalTok{ pd}

\NormalTok{    submission }\OperatorTok{=}\NormalTok{ pd.DataFrame(\{}\StringTok{"score"}\NormalTok{:confidence\_scores\})}
\NormalTok{    submission.to\_csv(save\_path, index\_label}\OperatorTok{=}\StringTok{"id"}\NormalTok{)}
\end{Highlighting}
\end{Shaded}

\subsection{Exercise 2 - Clustering}\label{exercise-2---clustering}

\textbf{{[}25 points{]}}

Clustering can be used to reveal structure between samples of data and
assign group membership to similar groups of samples. This exercise will
provide you with experience applying clustering algorithms and comparing
these techniques on various datasets to experience the pros and cons of
these approaches when the structure of the data being clustered varies.
For this exercise, we'll explore clustering in two dimensions to make
the results more tangible, but in practice these approaches can be
applied to any number of dimensions.

\emph{Note: For each set of plots across the five datasets, please
create subplots within a single figure (for example, when applying
DBSCAN - please show the clusters resulting from DBSCAN as a single
figure with one subplot for each dataset). This will make comparison
easier.}

\textbf{2.1. Run K-means and choose the number of clusters}. Five
datasets are provided for you below and the code to load them below.

\begin{itemize}
\tightlist
\item
  Scatterplot each dataset
\item
  For each dataset run the k-means algorithm for values of \(k\) ranging
  from 1 to 10 and for each plot the ``elbow curve'' where you plot
  dissimilarity in each case. Here, you can measure dissimilarity using
  the within-cluster sum-of-squares, which in sklean is known as
  ``inertia'' and can be accessed through the \texttt{inertia\_}
  attribute of a fit KMeans class instance.
\item
  For each dataset, where is the elbow in the curve of within-cluster
  sum-of-squares and why? Is the elbow always clearly visible? When it's
  not clear, you will have to use your judgment in terms of selecting a
  reasonable number of clusters for the data. \emph{There are also other
  metrics you can use to explore to measure the quality of cluster fit
  (but do not have to for this assignment) including the silhouette
  score, the Calinski-Harabasz index, and the Davies-Bouldin, to name a
  few within sklearn alone. However, assessing the quality of fit
  without ``preferred'' cluster assignments to compare against (that is,
  in a truly unsupervised manner) is challenging because measuring
  cluster fit quality is typically poorly-defined and doesn't generalize
  across all types of inter- and intra-cluster variation.}
\item
  Plot your clustered data (different color for each cluster assignment)
  for your best \(k\)-means fit determined from both the elbow curve and
  your judgment for each dataset and your inspection of the dataset.
\end{itemize}

\textbf{2.2. Apply DBSCAN}. Vary the \texttt{eps} and
\texttt{min\_samples} parameters to get as close as you can to having
the same number of clusters as your choices with K-means. The same code
plots as gray/black any points that were not assigned to clusters.

\textbf{2.3. Apply Spectral Clustering}. Select the same number of
clusters as selected by k-means.

\textbf{2.4. Comment on the strengths and weaknesses of each approach}.
In particular, mention:

\begin{itemize}
\tightlist
\item
  Which technique worked ``best'' and ``worst'' (as defined by matching
  how human intuition would cluster the data) on each dataset?
\item
  How much effort was required to get good clustering for each method
  (how much parameter tuning needed to be done)?
\end{itemize}

\begin{tcolorbox}[enhanced jigsaw, colframe=quarto-callout-note-color-frame, rightrule=.15mm, opacityback=0, breakable, bottomtitle=1mm, opacitybacktitle=0.6, left=2mm, colbacktitle=quarto-callout-note-color!10!white, arc=.35mm, title=\textcolor{quarto-callout-note-color}{\faInfo}\hspace{0.5em}{Important note}, toprule=.15mm, toptitle=1mm, titlerule=0mm, bottomrule=.15mm, coltitle=black, leftrule=.75mm, colback=white]

For these clustering plots in this question, do NOT include legends
indicating cluster assignment; instead, just make sure the cluster
assignments are clear from the plot (e.g.~different colors for each
cluster)

\end{tcolorbox}

Code is provided below for loading the datasets and for making plots
with the clusters as distinct colors

\begin{Shaded}
\begin{Highlighting}[]
\CommentTok{\#\#\#\#\#\#\#\#\#\#\#\#\#\#\#\#\#\#\#\#\#\#\#\#\#\#\#\#\#\#\#\#}
\CommentTok{\# Load the data}
\CommentTok{\#\#\#\#\#\#\#\#\#\#\#\#\#\#\#\#\#\#\#\#\#\#\#\#\#\#\#\#\#\#\#\#}
\ImportTok{import}\NormalTok{ os}
\ImportTok{import}\NormalTok{ numpy }\ImportTok{as}\NormalTok{ np}
\ImportTok{import}\NormalTok{ matplotlib.pyplot }\ImportTok{as}\NormalTok{ plt}
\ImportTok{from}\NormalTok{ sklearn.datasets }\ImportTok{import}\NormalTok{ make\_blobs, make\_moons}

\CommentTok{\# Create / load the datasets:}
\NormalTok{n\_samples }\OperatorTok{=} \DecValTok{1500}
\NormalTok{X0, \_ }\OperatorTok{=}\NormalTok{ make\_blobs(n\_samples}\OperatorTok{=}\NormalTok{n\_samples, centers}\OperatorTok{=}\DecValTok{2}\NormalTok{, n\_features}\OperatorTok{=}\DecValTok{2}\NormalTok{, random\_state}\OperatorTok{=}\DecValTok{0}\NormalTok{)}
\NormalTok{X1, \_ }\OperatorTok{=}\NormalTok{ make\_blobs(n\_samples}\OperatorTok{=}\NormalTok{n\_samples, centers}\OperatorTok{=}\DecValTok{5}\NormalTok{, n\_features}\OperatorTok{=}\DecValTok{2}\NormalTok{, random\_state}\OperatorTok{=}\DecValTok{0}\NormalTok{)}

\NormalTok{random\_state }\OperatorTok{=} \DecValTok{170}
\NormalTok{X, y }\OperatorTok{=}\NormalTok{ make\_blobs(n\_samples}\OperatorTok{=}\NormalTok{n\_samples, random\_state}\OperatorTok{=}\NormalTok{random\_state, cluster\_std}\OperatorTok{=}\FloatTok{1.3}\NormalTok{)}
\NormalTok{transformation }\OperatorTok{=}\NormalTok{ [[}\FloatTok{0.6}\NormalTok{, }\OperatorTok{{-}}\FloatTok{0.6}\NormalTok{], [}\OperatorTok{{-}}\FloatTok{0.2}\NormalTok{, }\FloatTok{0.8}\NormalTok{]]}
\NormalTok{X2 }\OperatorTok{=}\NormalTok{ np.dot(X, transformation)}
\NormalTok{X3, \_ }\OperatorTok{=}\NormalTok{ make\_blobs(n\_samples}\OperatorTok{=}\NormalTok{n\_samples, cluster\_std}\OperatorTok{=}\NormalTok{[}\FloatTok{1.0}\NormalTok{, }\FloatTok{2.5}\NormalTok{, }\FloatTok{0.5}\NormalTok{], random\_state}\OperatorTok{=}\NormalTok{random\_state)}
\NormalTok{X4, \_ }\OperatorTok{=}\NormalTok{ make\_moons(n\_samples}\OperatorTok{=}\NormalTok{n\_samples, noise}\OperatorTok{=}\FloatTok{.12}\NormalTok{)}

\NormalTok{X }\OperatorTok{=}\NormalTok{ [X0, X1, X2, X3, X4]}
\CommentTok{\# The datasets are X[i], where i ranges from 0 to 4}
\end{Highlighting}
\end{Shaded}

\begin{Shaded}
\begin{Highlighting}[]
\CommentTok{\#\#\#\#\#\#\#\#\#\#\#\#\#\#\#\#\#\#\#\#\#\#\#\#\#\#\#\#\#\#\#\#}
\CommentTok{\# Code to plot clusters}
\CommentTok{\#\#\#\#\#\#\#\#\#\#\#\#\#\#\#\#\#\#\#\#\#\#\#\#\#\#\#\#\#\#\#\#}
\KeywordTok{def}\NormalTok{ plot\_cluster(ax, data, cluster\_assignments):}
    \CommentTok{\textquotesingle{}\textquotesingle{}\textquotesingle{}Plot two{-}dimensional data clusters}
\CommentTok{    }
\CommentTok{    Parameters}
\CommentTok{    {-}{-}{-}{-}{-}{-}{-}{-}{-}{-}}
\CommentTok{    ax : matplotlib axis}
\CommentTok{        Axis to plot on}
\CommentTok{    data : list or numpy array of size [N x 2] }
\CommentTok{        Clustered data}
\CommentTok{    cluster\_assignments : list or numpy array [N]}
\CommentTok{        Cluster assignments for each point in data}

\CommentTok{    \textquotesingle{}\textquotesingle{}\textquotesingle{}}
\NormalTok{    clusters }\OperatorTok{=}\NormalTok{ np.unique(cluster\_assignments)}
\NormalTok{    n\_clusters }\OperatorTok{=} \BuiltInTok{len}\NormalTok{(clusters)}
    \ControlFlowTok{for}\NormalTok{ ca }\KeywordTok{in}\NormalTok{ clusters:}
\NormalTok{        kwargs }\OperatorTok{=}\NormalTok{ \{\}}
        \ControlFlowTok{if}\NormalTok{ ca }\OperatorTok{==} \OperatorTok{{-}}\DecValTok{1}\NormalTok{:}
            \CommentTok{\# if samples are not assigned to a cluster (have a cluster assignment of {-}1, color them gray)}
\NormalTok{            kwargs }\OperatorTok{=}\NormalTok{ \{}\StringTok{\textquotesingle{}color\textquotesingle{}}\NormalTok{:}\StringTok{\textquotesingle{}gray\textquotesingle{}}\NormalTok{\}}
\NormalTok{            n\_clusters }\OperatorTok{=}\NormalTok{ n\_clusters }\OperatorTok{{-}} \DecValTok{1}
\NormalTok{        ax.scatter(data[cluster\_assignments}\OperatorTok{==}\NormalTok{ca, }\DecValTok{0}\NormalTok{], data[cluster\_assignments}\OperatorTok{==}\NormalTok{ca, }\DecValTok{1}\NormalTok{],s}\OperatorTok{=}\DecValTok{5}\NormalTok{,alpha}\OperatorTok{=}\FloatTok{0.5}\NormalTok{, }\OperatorTok{**}\NormalTok{kwargs)}
\NormalTok{        ax.set\_xlabel(}\StringTok{\textquotesingle{}feature 1\textquotesingle{}}\NormalTok{)}
\NormalTok{        ax.set\_ylabel(}\StringTok{\textquotesingle{}feature 2\textquotesingle{}}\NormalTok{)}
\NormalTok{        ax.set\_title(}\SpecialStringTok{f\textquotesingle{}No. Clusters = }\SpecialCharTok{\{}\NormalTok{n\_clusters}\SpecialCharTok{\}}\SpecialStringTok{\textquotesingle{}}\NormalTok{)}
\NormalTok{        ax.axis(}\StringTok{\textquotesingle{}equal\textquotesingle{}}\NormalTok{)}
\end{Highlighting}
\end{Shaded}

\subsection{Exercise 3 - Dimensionality reduction and visualization of
digits with PCA and
t-SNE}\label{exercise-3---dimensionality-reduction-and-visualization-of-digits-with-pca-and-t-sne}

\textbf{{[}25 points{]}}

\textbf{3.1.} Reduce the dimensionality of the data with PCA for data
visualization. Load the \texttt{scikit-learn} digits dataset (code
provided to do this below). Consider whether any preprocessing may need
to be applied (do the data need to be normalized?). Apply PCA and reduce
the data (with the associated cluster labels 0-9) into a 2-dimensional
space. Plot the data with labels in this two dimensional space (labels
can be colors, shapes, or using the actual numbers to represent the data
- definitely include a legend in your plot).

\textbf{3.2.} Create a plot showing the cumulative fraction of variance
explained as you incorporate from \(1\) through all \(D\) principal
components of the data (where \(D\) is the dimensionality of the data).

\begin{itemize}
\tightlist
\item
  What fraction of variance in the data is UNEXPLAINED by the first two
  principal components of the data?
\item
  Briefly comment on how this may impact how well-clustered the data
  are. \emph{You can use the \texttt{explained\_variance\_} attribute of
  the PCA module in \texttt{scikit-learn} to assist with this question}
\end{itemize}

\textbf{3.3.} Reduce the dimensionality of the data with t-SNE for data
visualization. T-distributed stochastic neighborhood embedding (t-SNE)
is a nonlinear dimensionality reduction technique that is particularly
adept at embedding the data into lower 2 or 3 dimensional spaces. Apply
t-SNE using the \texttt{scikit-learn} implementation to the digits
dataset and plot it in 2-dimensions (with associated cluster labels
0-9). You may need to adjust the parameters to get acceptable
performance. You can read more about how to use t-SNE effectively
\href{https://distill.pub/2016/misread-tsne/}{here}.

\textbf{3.4.} Briefy compare/contrast the performance of these two
techniques.

\begin{itemize}
\tightlist
\item
  Which seemed to cluster the data best and why?
\item
  Notice that while t-SNE has a \texttt{fit} method and a
  \texttt{fit\_transform} method, these methods are actually identical,
  and there is no \texttt{transform} method. Why is this? What
  implications does this imply for using this method?
\end{itemize}

\emph{Note: Remember that you typically will not have labels available
in most problems.}

Code is provided for loading the data below.

\begin{Shaded}
\begin{Highlighting}[]
\CommentTok{\#\#\#\#\#\#\#\#\#\#\#\#\#\#\#\#\#\#\#\#\#\#\#\#\#\#\#\#\#\#\#\#}
\CommentTok{\# Load the data}
\CommentTok{\#\#\#\#\#\#\#\#\#\#\#\#\#\#\#\#\#\#\#\#\#\#\#\#\#\#\#\#\#\#\#\#}
\ImportTok{from}\NormalTok{ sklearn }\ImportTok{import}\NormalTok{ datasets}
\ImportTok{from}\NormalTok{ sklearn.decomposition }\ImportTok{import}\NormalTok{ PCA}
\ImportTok{from}\NormalTok{ sklearn.manifold }\ImportTok{import}\NormalTok{ TSNE}

\CommentTok{\# load dataset}
\NormalTok{digits }\OperatorTok{=}\NormalTok{ datasets.load\_digits()}
\NormalTok{n\_sample }\OperatorTok{=}\NormalTok{ digits.target.shape[}\DecValTok{0}\NormalTok{]}
\NormalTok{n\_feature }\OperatorTok{=}\NormalTok{ digits.images.shape[}\DecValTok{1}\NormalTok{] }\OperatorTok{*}\NormalTok{ digits.images.shape[}\DecValTok{2}\NormalTok{]}
\NormalTok{X\_digits }\OperatorTok{=}\NormalTok{ np.zeros((n\_sample, n\_feature))}
\ControlFlowTok{for}\NormalTok{ i }\KeywordTok{in} \BuiltInTok{range}\NormalTok{(n\_sample):}
\NormalTok{    X\_digits[i, :] }\OperatorTok{=}\NormalTok{ digits.images[i, :, :].flatten()}
\NormalTok{y\_digits }\OperatorTok{=}\NormalTok{ digits.target}
\end{Highlighting}
\end{Shaded}





\end{document}
